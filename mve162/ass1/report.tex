\documentclass{article}
\usepackage[T1]{fontenc}
\usepackage[utf8]{inputenc}
\usepackage[swedish]{babel}
\usepackage[backend=biber]{biblatex}
\usepackage{amsmath, amsfonts, amsthm, mathtools}
\usepackage[hidelinks]{hyperref}
\usepackage{cancel}
\usepackage{siunitx}

\bibliography{sources}
\sisetup{
	round-mode=figures,
	round-precision=3,
	}
\DeclareSIUnit\st{st}

\newtheorem{theorem}{Theorem}
\newtheorem{definition}{Definition}
\newtheorem{lemma}[theorem]{Lemma}
\newtheorem{proposition}{Proposition}
\newtheorem{corollary}{Corollary}[theorem]

\DeclareMathOperator{\diag}{diag}

\DeclarePairedDelimiter\abs{\lvert}{\rvert}
\newcommand{\norm}[1]{\left\lVert#1\right\rVert}

\makeatletter
\let\oldabs\abs
\def\abs{\@ifstar{\oldabs}{\oldabs*}}
% \let\oldnorm\norm
% \def\norm{\@ifstar{\oldnorm}{\oldnorm*}}
\makeatother

\title{MVE162}
\author{Axel Forsman}

\begin{document}
% \maketitle

Analysera Coronavirus spridning i Sverige.
Vi betraktar susceptible-infected-removed (SIR) modellen
\begin{equation}\label{eq:sir}
	\begin{alignedat}{4}
		& \frac{dS}{dt} &=& -&&rSI && \\
		& \frac{dI}{dt} &=& &&rSI &-& \gamma I \\
		& \frac{dR}{dt} &=& && &&\gamma I
	\end{alignedat}
\end{equation}
där $S(t), I(t), R(t)$ betecknar antalet motagliga, infekterade
respektive borttagna individer efter $t \in [0, \infty)$ veckor,
$S, I, R \ge 0$.
Låt $N = \SI{10327589}{\st}$ vara antalet invånare i Sverige
(enligt mätningar från 2020) \autocite{population}.
Begynnelsevärden läts spegla situationen just innan
första smittspridningen i Sverige,
med $S(0) = N, R(0) = 0$ och $I(0) = I_0$,
där $I_0 = 1$ är det lilla antalet från början importerade infekterade -
som vi kommer se har det inte för stor påverkan på modellen. % TODO

% Question 1

Då de två första ekvationerna i~\eqref{eq:sir}
är oberoende av $R$
kan vi betrakta dem separat
$$ \frac{dI}{dt} \overset{\text{chain}}{\underset{\text{rule}}=} \frac{dI}{dS} \frac{dS}{dt}
\implies \Phi(S) \coloneqq \frac{dI}{dS} \overset{\eqref{eq:sir}}= \frac{rSI - \gamma I}{-rSI} = \frac\gamma{rS} - 1 $$
under antagandet att $S \ne 0$.
Integration av $\Phi$ ger ett explicit uttryck för $I(S)$,
$$ I(S) = \int \Phi(\sigma) \, d\sigma = \frac\gamma r \ln S - S + C $$
där $C$ är integrationskonstanten som finnes genom
insättning av begynnelsevärdena $I(S) \vert_{S=N} = I_0$.
Vi ser att $I(S) \to 0, S \to 1$ och $I(S) \to -\infty, S \to \infty$.
För att undersöka eventuella extrempunkter
noterar vi att $\Phi(S) = 0 \Leftrightarrow S = \gamma / r \eqqcolon \rho$,
vilket är en maximipunkt enligt gränsvärdesuppförandet.
% TODO Add plot
Vidare är $\Phi(S) > 0 \quad \forall S < \rho$,
vilket sammantaget med faktumet att
$\frac{dS}{dt} < 0 \quad \forall t$
gör att $I(t)$ är monotont avtagande för $t : S(t) \le \rho$ -
varför $\rho$ är den övre gräns på antalet motagliga
för att epidemin med tiden ska dö ut.

Med veckor som tidsenhet.
\cite{folkhalso}

\printbibliography

\end{document}
