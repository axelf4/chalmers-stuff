\documentclass{article}
\usepackage[T1]{fontenc}
\usepackage[utf8]{inputenc}
\usepackage[swedish]{babel}
\usepackage[backend=biber]{biblatex}
\usepackage{amsmath, amsfonts, amsthm, mathtools}
\usepackage[hidelinks]{hyperref}
\usepackage{cancel}
\usepackage{siunitx}
\usepackage{pgfplots, pgfplotstable}
\usepackage[useregional]{datetime2}
\usepackage{listings}
\usepackage{lmodern, microtype}

\pgfplotsset{compat=1.15}
\usepgfplotslibrary{dateplot,colorbrewer}
\pgfplotsset{cycle list/Dark2,colormap/GnBu}
\bibliography{sources}
\sisetup{
	range-phrase = \,till\,,
	round-mode=figures,
	round-precision=3,
	}
\DeclareSIUnit\st{st}
\lstset{
	breaklines=true,
	postbreak=\mbox{$\hookrightarrow$\space},
	breakindent = 5pt,
	breakautoindent = false,
}

\DeclareMathOperator*{\argmin}{arg\,min}
\DeclarePairedDelimiter\abs{\lvert}{\rvert}
\newcommand{\norm}[1]{\left\lVert#1\right\rVert}

\makeatletter
\def\fps@figure{hbtp}
\def\fps@table{hbtp}

\let\oldabs\abs
\def\abs{\@ifstar{\oldabs}{\oldabs*}}
% \let\oldnorm\norm
% \def\norm{\@ifstar{\oldnorm}{\oldnorm*}}
\makeatother

\newcommand\covid{COVID\nobreakdash-19}

\title{Två modeller för \covid{} epidemin i Sverige}
\author{Axel Forsman}

\begin{document}
\maketitle

\begin{abstract}
	Två enkla ordinära differentialekvationsmodeller
	för \covid{} epidemin i Sverige presenteras,
	den andra utökad från den första för att omfatta
	Sveriges isolationspolitik för personer 70 år och äldre.
	Statistik från SCB används för att estimera parameterna
	och deras lösningar analyseras.
	Resultatet är att en väldigt stor andel av befolkningen
	(\SI{0.9947}\percent) kommer smittas innan epidemin dör ut,
	och, i den andra modellen, att det gått att mitigera
	om alla över 20 år isolerats.
\end{abstract}

\section{SIR-modellen}

För att analysera \covid:s spridning i Sverige
betraktar vi susceptible-infected-removed (SIR) modellen
\begin{equation}\label{eq:sir}
	\begin{alignedat}{4}
		& \frac{dS}{dt} &=& -&&rSI && \\
		& \frac{dI}{dt} &=& &&rSI &-& \gamma I \\
		& \frac{dR}{dt} &=& && &&\gamma I
	\end{alignedat}
\end{equation}
där $S(t), I(t), R(t) \ge 0$ betecknar antalet mottagliga, infekterade
respektive borttagna (antingen återhämtade eller bortgångna)
individer efter $t \in \mathbb R^+$ veckor.
\def\N{10327589}
Låt $N = \SI{10327589}\st$ vara antalet invånare i Sverige
(enligt mätningar från 2020) \autocite{population}.
Begynnelsevärden läts spegla situationen just innan
första smittspridningen i Sverige,
med $S(0) = N, R(0) = 0$ och $I(0) = I_0$,
där $I_0 = 1$ är det lilla antalet från början importerade infekterade.

% Question 1
Då de två första ekvationerna i~\eqref{eq:sir}
är oberoende av $R$
kan vi betrakta dem separat
$$ \frac{dI}{dt} \overset{\text{chain}}{\underset{\text{rule}}=} \frac{dI}{dS} \frac{dS}{dt}
\implies \Phi(S) \coloneqq \frac{dI}{dS} \overset{\eqref{eq:sir}}= \frac{rSI - \gamma I}{-rSI} = \frac\gamma{rS} - 1 $$
under antagandet att $S \ne 0$.
Integration av $\Phi$ ger ett explicit uttryck för $I(S)$,
$$ I(S) = \int \Phi(\sigma) \, d\sigma = \frac\gamma r \ln S - S + C $$
där $C$ är integrationskonstanten som finnes genom
insättning av begynnelsevärdena $I(S) \vert_{S=N} = I_0$,
vilket ger $C = I_0 + N - \frac\gamma r \ln N$.
% TODO Add plot
Vi ser att $I(S) \to C - 1, S \to 1$ och $I(S) \to -\infty, S \to \infty$.
För att undersöka eventuella extrempunkter
noterar vi att $\Phi(S) = 0 \Leftrightarrow S = \gamma / r \eqqcolon \rho$,
vilket är en maximipunkt enligt gränsvärdesuppförandet.
Vidare är $\Phi(S) > 0 \quad \forall S < \rho$,
vilket sammantaget med faktumet att
$\frac{dS}{dt} < 0 \quad \forall t$
gör att $I(t)$ är monotont avtagande för $t : S(t) \le \rho$ -
varför $\rho$ är den övre gräns på antalet mottagliga
för att epidemin med tiden ska dö ut.

% Question 2
Det är vedertaget att \covid{} har den infekterade antingen
friskna till eller avlida ungefär fyra veckor efter insjuknandet -
detta bestämmer värdet av parametern $\gamma$,
takten med vilken individer lämnar $I$ till $R$,
$$ \gamma = \frac14 $$
För att estimera parametern $r$ görs regression mot data över
antal laboratoriebekräftade smittofall per dag i Sverige
\cite{folkhalso}.
Dessutom kommer vi anta att de registrerade fallen endast representerar
\SIrange{1}{5}{\percent} av det verkliga värdet.
Låt $t=0$ representera \DTMdate{2020-02-04} då första infektionen inträffade.
$$ r^* \coloneqq \argmin_r \sum_{t \in T_q} (S_r(t) - S_{\text{actual}}(t))^2 $$
där
$$ T_q = \left\{\text{2020-03-12} + n \cdot 1 \text{\,vecka} \le \text{2020-04-29} \vert n \in \mathbb N \cup \{0\} \right\} $$
Från och med den 12:e mars ändrades metodiken för vilka individer
som testades till att bara inkludera inlagda personer från riskgrupper
och sjukhuspersonal med symptom,
varför endast data efter det datumet kommer beaktas.
Vi får att
$$ r^* = \num{2.3235e-07} $$

\pgfplotstableset{
	create on use/cumulative cases/.style={
		create col/expr={\pgfmathaccuma + \thisrow{Fall}}
		}
	}

\begin{figure}
	\centering
	\begin{tikzpicture}
		\begin{axis}[
				date coordinates in=x,
				date ZERO=2020-03-12,
				xticklabel=\month-\day,
				xlabel=Tid, ylabel=Fall,
				]
				\addplot table [x=Datum,y=cumulative cases,col sep=comma] {FolkhalsomyndighetenCovid19.csv};
				% \addplot table [x=Datum,y=Fall,col sep=comma] {FolkhalsomyndighetenCovid19.csv};
		\end{axis}
	\end{tikzpicture}
	\caption{Kumulativ graf över laboratoriebekräftade fall av \covid{} i Sverige.
	\autocite{folkhalso} \label{fig:cum_cases}}
\end{figure}

% Question 4.
För dessa parametrar blir värdet av det tidigare nämnda $\rho$
\def\rhoVal{1965254.303907}
$$ \rho = \frac\gamma r = \num{\rhoVal} < N = \num{\N} $$
det vill säga,
enligt modellen kommer antalet fall per dag öka tills
invånarantalet når $\rho$, för att sedan avta.
Den numeriska lösningen för parametrarna i fråga är redovisad i figur~\ref{fig:sir_solution}
med de augmenterade värdena för antalet smittofall.
Vi ser att $S(t)$ asymptotiskt går mot 0 alltså att alla blir infekterade.

\pgfplotstableset{
	create on use/mod t/.style={
		create col/expr={\pgfplotstablerow / 7}
		}
	}
\pgfplotstableset{
	create on use/real S/.style={
		create col/expr accum={\pgfmathaccuma - 1 / 0.05 * \thisrow{Fall}}{\N}
		}
	}
\begin{figure}
	\centering
	\begin{tikzpicture}
		\begin{axis}[
				width = 0.8\linewidth,
				xlabel=$t$, ylabel=Antal individer,
				xmin=0, xmax=25,
				legend style={at={(0.05,0.65)},anchor=west},
				extra y ticks = \rhoVal, extra y tick labels = {$\rho$},
				extra y tick style = {ticklabel pos=right, grid = major},
				]
				\addplot table [x=t,y=S,col sep=comma] {sirResult.csv};
				\addplot table [x=t,y=I,col sep=comma] {sirResult.csv};
				\addplot table [x=t,y=R,col sep=comma] {sirResult.csv};
				\addplot table [x=mod t,y=real S,col sep=comma] {FolkhalsomyndighetenCovid19.csv};
				\legend{$S(t)$,$I(t)$,$R(t)$,`Verklig' $S(t)$}
		\end{axis}
	\end{tikzpicture}
	\caption{Numerisk lösning till \eqref{eq:sir} med uppskattade värden på parametrarna $r$ och $\gamma$,
	tillsammans med de estimerade riktiga värdena på $S(t)$
	(dvs. $N - \frac1{\SI5\percent} \text{TotaltAntalFall}$). \label{fig:sir_solution}}
\end{figure}

Efter epidemin kommer $I(t) \approx 0$,
och då $S(t) + I(t) + R(t) = N$
räcker det att veta $S(t)$,
varför vi för att undersöka hur variationer av parametrarna påverkar lösningen
betraktar andelen som blir infekterade för olika värden på $r$ och $\gamma$,
se figur~\ref{fig:param_variation}.
Vi ser att andelen ökar med $r$ - fler blir infekterade -
och minskar med $\gamma$ - individer smittar under längre tid.
Att multiplicera $r$ och $\gamma$ med samma tal innebär att
högerledet i \eqref{eq:sir} skalas om;
då systemet är autonomt leder detta endast till att
tidsskalan ändras,
vilket uppenbarar sig som linjer i $r$\nobreakdash-$\gamma$\nobreakdash-planet
med konstant andel.
Dessa linjer borde mötas i origo då var punkt kan skalas
så att komma närmre origo.

\begin{figure}
	\centering
	\begin{tikzpicture}
		\begin{axis}[
				colorbar,
				view={0}{90},
				mesh/rows=30,
				xlabel=$r$, ylabel=$\gamma$,
				]
				\addplot3[surf] table[col sep=comma] {paramHeatmap.txt};
				\node[mark size=3pt, olive] at (axis cs:2.3235e-07, 0.25) {\pgfuseplotmark{*}};
		\end{axis}
	\end{tikzpicture}
	\caption{Andel av befolkningen som blir infekterad av epidemin
	för olika värden av parametrarna.
	Markören visar de estimerade parametrarna. \label{fig:param_variation}}
\end{figure}

% Question 3 & 5.
Hittills är det totala antalet rapporterade infekterade 20302 stycken,
och det totala antalet dödsfall, 2449 stycken \autocite{folkhalso}.
Därför är en grov uppskattning av dödligheten till följd av \covid{},
$$ D_\% = \frac{2449}{\frac1{\SI5{\percent}} 20302} = \SI{0.6031}{\percent} $$
vilket bortser från hysteres orsakad av att flera av de nyligen infekterade
förväntas avlida i framtiden,
men är ändå i närheten av WHO:s uppskattning på \SIrange12{\percent} \autocite{faq}.
Då, enligt ovan, alla så småningom blir infekterade är också detta
andelen döda personer vid slutet av epidemin.

\section{Utökad SIR-modell för isolationspolitiken}

Sveriges politik om att isolera personer 70 år och äldre
ämnar sig bättre åt en lite annorlunda modell,
där $S$ och $R$ klasserna delats upp i $S_1, R_1$ och $S_2, R_2$
för de inte isolerats respektive resten,
med två olika infektionstakter $r_1, r_2$.
Nödvändigtvis, är då $r_2 < r_1$.
\begin{equation}\label{eq:ssirr}
	\begin{alignedat}{4}
		& \frac{dS_1}{dt} &=& &-r_1&S_1I && \\
		& \frac{dS_2}{dt} &=& &-r_2&S_2I && \\
		& \frac{dI}{dt} &=& &(r_1S_1 + r_2S_2)&I &-\gamma& I \\
		& \frac{dR_1}{dt} &=& && &\frac{r_1}{r_1 + r_2} \gamma& I \\
		& \frac{dR_2}{dt} &=& && &\frac{r_2}{r_1 + r_2} \gamma& I
	\end{alignedat}
\end{equation}
Åldersdistributionen i Sverige given av SCB är redovisad i figur~\ref{fig:pop_dist}.
Från den utläses att andelen personer av ålder $70+$ är
$$ p_{70+} = \SI{0.1477}\percent $$
vilket ger begynnelsevärdena $S_1(0) = (1 - p_{70+}) N, \, S_2(0) = p_{70+} N$,
med resten liknande de i \eqref{eq:sir}.
Under antagandet att vårt tidigare deriverade $r^*$ var optimalt
måste vi ha
$$ \frac{dS_1}{dt} + \frac{dS_2}{dt} = -r^* (S_1 + S_2) I \implies r_1 = r^* + (r^* - r_2) \frac{S_2}{S_1} $$
varför det räcker att estimera $r_2$, vilket görs på samma sätt som tidigare
så att andelen sjukdomsfall i åldersgruppen $\le 70$,
$p_{\le 70}^\text{sjukdomsfall} = \SI{0.6391}\percent$, se figur~\ref{fig:casesPerAge},
överensstämmer.
Vi får då
$$ r_1 = \num{1.4484e-07} \quad \text{och} \quad r_2 = \num{2.5441e-08} $$

\begin{figure}
	\centering
	\begin{tikzpicture}
		\begin{axis}[
				ybar interval,
				symbolic x coords={0-4, 5-9, 10-14, 15-19, 20-24, 25-29, 30-34, 35-39, 40-44, 45-49, 50-54, 55-59, 60-64, 65-69, 70-74, 75-79, 80-84, 85-89, 90-94, 95-99, 100+},
				xtick=data,
				xticklabel style = {rotate=45, anchor=north east},
				xlabel = Ålder, ylabel = Antal individer,
				width = \textwidth, height=0.5\textwidth,
				]
				\addplot table[x=Age,y=Individuals,col sep=comma] {populationDist.csv};
		\end{axis}
	\end{tikzpicture}
	\caption{Åldersdistributionen i Sverige år 2019. \autocite{populationDist} \label{fig:pop_dist}}
\end{figure}

Med takten då individer i isolation blir smittade estimerad,
kan vi sedan ändra andelen av befolkningen som är isolerad separat.
$R_1(t)$ och $R_2(t)$ är då antalet borttagna individer som
inte varit isolerade respektive varit.
Med statistik över antalet sjukdomsfall och dödsfall,
se tabell~\ref{fig:casesPerAge},
kan vi uppskatta dödsrisken för smittade av olika åldersgrupper.
Figur~\ref{fig:isolation_values} visar resultatet.
Vi ser att om istället alla 30 år och äldre isolerats
så hålls dödstalet väldigt lågt.

\begin{figure}
	\centering
	\begin{tikzpicture}
		\begin{axis}[
				ybar interval,
				symbolic x coords={0-9, 10-19, 20-29, 30-39, 40-49, 50-59, 60-69, 70-79, 80-89, 90+},
				xtick=data,
				xticklabel style = {rotate=45, anchor=north east},
				xlabel = Ålder, ylabel = Antal fall,
				width = \textwidth, height=0.5\textwidth,
				legend style={at={(0.05,0.95)},anchor=north west},
				]
				\addplot table[x=Age,y=Infections,col sep=comma] {casesPerAge.csv};
				\addplot table[x=Age,y=Diseased,col sep=comma] {casesPerAge.csv};
				\legend{Sjukdomsfall, Dödsfall}
		\end{axis}
	\end{tikzpicture}
	\caption{Sjukdomsfall och dödsfall per åldersgrupp. \autocite{folkhalso} \label{fig:casesPerAge}}
\end{figure}

\begin{figure}
	\centering
	\begin{tikzpicture}
		\begin{axis}[
				ybar interval,
				xlabel = Antal åldersgrupper som inte är isolerade, ylabel = Antal fall,
				legend style={at={(0.05,0.95)},anchor=north west},
				]
				\addplot table[x=i,y=Infected,col sep=comma] {isolationVals.csv};
				\addplot table[x=i,y=Dead,col sep=comma] {isolationVals.csv};
				\legend{Sjukdomsfall, Dödsfall}
		\end{axis}
	\end{tikzpicture}
	\caption{Totala antalet infekterade och döda i modell \eqref{eq:ssirr} för olika
	antal åldersgrupper i isolation.
	Det vill säga, till exempel 1 korresponderar med att alla förutom 0-10 år är isolerade.
	\label{fig:isolation_values}}
\end{figure}

\vspace{2em}

Vi har sett två modeller för \covid{} epidemin i Sverige.
Graden med vilken de representerar verkligheten är dock diskuterbar.
Åtminstone ett exempel på förbättringsområde är vad som händer
efter att infektionsfallen ökat dramatiskt -
man kan tänka sig att det skulle leda ökad isolation.
Och förhoppningsvis fortsätter dessa modeller vara just modeller
och inte profetior.

\appendix

\section{MATLAB kod}
\lstinputlisting[language=MATLAB]{ass1.m}

\printbibliography

\end{document}
