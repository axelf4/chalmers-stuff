\documentclass{article}
\usepackage[T1]{fontenc}
\usepackage[utf8]{inputenc}
\usepackage[swedish]{babel}
\usepackage{amsmath, amssymb, mathtools}
\usepackage{tikz}
\usepackage{enumitem}
\usepackage{siunitx}
\usepackage{ifthen}
\usepackage{listings}

\usetikzlibrary{arrows, decorations.markings, calc, shapes.geometric}

\DeclareMathOperator\rowSums{rowSums}

\begin{document}

\section{Question 1}
We assume the number of customers each day at a taxidermist's shop is
Poisson($\lambda$) distributed, where $\lambda$ is the expected number of daily customers.
The shop receives 2, 0, and 1 customers on Monday, Tuesday, and Wednesday,
respectively. We would like to predict how many customers there will
be on Thursday.
\begin{enumerate}[label=(\alph*)]
    \item Based on previous experiences in similar shops, we assume the prior
        probability distribution for $\lambda$ is Gamma(3, 4). Compute the posterior
        probability distribution for $\lambda$ given the data above. Also, given the
        data, compute the predicted probabilities for getting zero, 1, 2, 3, or
        4 or more customers on Thursday.
    \item Simulate in R, $N$ = 1.000.000 variables $\lambda_1, ..., \lambda_N$ from the Gamma(3, 4)
        distribution. For each $\lambda_i$ simulate $y_i$ from the Poisson($\lambda_i$) distribution,
        and select those $\lambda_i$ with a simulated $y_i$ value equal to 2.
        Compute theoretically the expected proportion you would select, and
        check with your simulations. Compute also theoretically the distribution
        of the $\lambda_i$ you would select, and compare with the results of
        your simulation.
    \item Starting with the sample above of size one million, simulate for each
        $\lambda_i$ values $y_{i1}, y_{i2}$ and $y_{i3}$ that are Poisson($\lambda_i$) distributed. 
        By comparing the simulated values to 2, 0, and 1, respectively, obtain a
        sample which can be used to obtain an approximate answer to the
        question you answered in (a). Compare your approximate answer to
        your answer from (a).
    \item The answer in (a) can also be computed using numerical integration,
        similar to the example in Section 1.6 of the Lecture Notes. Make
        such a computation, and compare with your results from (a).
\end{enumerate}

\subsection{Part (a)}
\subsection{Part (d)}
Compute $\pi(Y \mid X)$ by numerical integration.
Throwing $\lambda$ into the mix and integrating it out:
\begin{align*}
	\pi(Y \mid X) &= \int_{\Lambda = \mathbb R^+} \pi(Y, \lambda \mid X) \, d\lambda \\
	&= \int_\Lambda \pi(Y \mid X) \pi(\lambda \mid X) \, d\lambda
\end{align*}
where we have used $\pi(Y, \lambda \mid X) = \pi(Y \mid X, \lambda) \pi(\lambda \mid X)$,
and the fact that the prediction given the parameter is independent of the data.
Using Bayes' theorem and the law of total probability
$$ \pi(\lambda \mid X) = \frac{\pi(X \mid \lambda) \pi(\lambda)}{\int_\lambda \pi(X \mid \lambda) \pi(\lambda) \, d\lambda} $$

\section{Question \num{3.52} in Dobrow.}
With the rule variation that you have to roll the exact number to reach
the final square.
\begin{enumerate}[label=(\alph*)]
	\item Find the expected length of the game.
	\item Assume that the player is on square 6.
		Find the probability that they will find themselves on square 3 before finishing the game.

		Modify $P$ to make $3$ an absorbing state and repeat the above.
\end{enumerate}

\subsection{(a)}

\begin{center}
	\def\ladderwidth{1.5mm}
	\def\ladderstep{1.5mm}
	\begin{tikzpicture}[
			state/.style={ draw, regular polygon, regular polygon sides=4, fill=gray!15 },
			every edge/.style={ draw, ->,>=stealth', auto, },
			chute/.style={ top color=gray!50, bottom color=gray!50,
			middle color=gray!10, draw=gray!50!black, line cap=round,
			draw opacity=0.3, fill opacity=0.5,
			line join=round,line width=1pt,},
			ladder/.style={decorate,decoration={
				markings,
				mark=between positions {1/#1/2} and {-1/#1/2} step {1/#1} with {
					\draw[draw=gray, draw opacity=0.5,
			line cap=round, line join=round, line width=2pt,]
					(0,-\ladderwidth) -- (0,\ladderwidth)
					(\ladderstep,-\ladderwidth) -- (-\ladderstep,-\ladderwidth)
					(\ladderstep,\ladderwidth) -- (-\ladderstep,\ladderwidth);
					},
					},
					},
					ladder auto/.style={
						to path={
							let \p1=($(\tikztostart) - (\tikztotarget)$), \n1={veclen(\x1,\y1)} in
							\pgfextra{
								\pgfmathsetmacro{\bars}{int(\n1/\ladderstep/2)+1}
								\pgfinterruptpath
								\draw[ladder=\bars] (\tikztostart) -- (\tikztotarget);
								\endpgfinterruptpath
								}
								},
								},
								chute auto/.style={
									to path={
										let
										\p1=([xshift=\ladderwidth]\tikztostart),
										\p2=([xshift=-\ladderwidth]\tikztostart),
										\p3=([xshift=\ladderwidth]\tikztotarget),
										\p4=([xshift=-\ladderwidth]\tikztotarget),
										\p5=($(\p1)!.4!(\p3)$),
										\p6=($(\p2)!.4!(\p4)$)
										in
										\pgfextra{
											\pgfinterruptpath
											\path[chute]
											(\p1) sin (\p5) cos (\p3) --
											(\p4) sin (\p6) cos (\p2) -- cycle;
											\endpgfinterruptpath
											}
											},
											},
											]
											\foreach \x in {0,...,2}
											\foreach \y in {0,...,2} {
												\pgfmathtruncatemacro\label{1 + \x + 3 * \y}
												\ifthenelse{1 = \y}{\def\xp{4 - 2 * \x}}{\def\xp{2 * \x}}
												\ifthenelse{\NOT 1 = \x}{
													\node[state] (\label) at (\xp, 2 * \y) {\label};
													}{
														\node[state, draw opacity=0.3, fill opacity=0.3] (\label) at (\xp, 2 * \y) {\label};
														}
														}

		\draw[chute auto] (3) to (5) (4) to (8);
		\draw[ladder auto] (2) to (7);

		\draw (1) edge[bend right] node[below] {\tt 2/4} (3);
		\draw (1) edge node[below] {\tt 1/4} (4);
		\draw (1) edge[bend left=70] node{\tt 1/4} (7);

		\draw (3) edge[loop right] node{\tt 1/4} (3);
		\draw (3) edge[bend right=70] node[right] {\tt 1/4} (4);
		\draw (3) edge[bend left] node{\tt 1/4} (6);
		\draw (3) edge[bend right] node[above] {\tt 1/4} (7);

		\draw (4) edge[loop above] node{\tt 1/4} (4);
		\draw (4) edge node{\tt 1/4} (3);
		\draw (4) edge[bend left] node[below left] {\tt 1/4} (6);
		\draw (4) edge[bend left] node{\tt 1/4} (7);

		\draw (6) edge[loop below] node{\tt 1/4} (6);
		\draw (6) edge node[below] {\tt 1/4} (4);
		\draw (6) edge[bend left] node{\tt 1/4} (7);
		\draw (6) edge node{\tt 1/4} (9);

		\draw (7) edge[loop above] node{\tt 2/4} (7);
		\draw (7) edge[bend left] node{\tt 1/4} (9);
		\draw (7) edge[bend left] node{\tt 1/4} (4);

		\draw (9) edge[loop right] node{\tt 1} (9);
	\end{tikzpicture}
\end{center}
The transition matrix is
\begin{equation}
	P \coloneqq \bordermatrix{~ & 1 & 3 & 4 & 6 & 7 & 9 \cr
		1 & 2/4 & 1/4 & 0 & 0 & 1/4 & 0 \cr
		3 & 0 & 1/4 & 1/4 & 1/4 & 1/4 & 0 \cr
		4 & 0 & 1/4 & 1/4 & 1/4 & 1/4 & 0 \cr
		6 & 0 & 0 & 1/4 & 1/4 & 1/4 & 1/4 \cr
		7 & 0 & 0 & 1/4 & 0 & 2/4 & 1/4 \cr
		9 & 0 & 0 & 0 & 0 & 0 & 1 \cr
		}
\end{equation}
This gives
$$ F \coloneqq (I - Q)^{-1} = \left(\begin{pmatrix}
		1 & 0 & 0 & 0 & 0 \\
		0 & 1 & 0 & 0 & 0 \\
		0 & 0 & 1 & 0 & 0 \\
		0 & 0 & 0 & 1 & 0 \\
		0 & 0 & 0 & 0 & 1 \\
\end{pmatrix} - \begin{pmatrix}
		2/4 & 1/4 & 0 & 0 & 1/4 \\
		0 & 1/4 & 1/4 & 1/4 & 1/4 \\
		0 & 1/4 & 1/4 & 1/4 & 1/4 \\
		0 & 0 & 1/4 & 1/4 & 1/4 \\
		0 & 0 & 1/4 & 0 & 2/4 \\
\end{pmatrix}\right)^{-1}
	= \begin{pmatrix}
		\cdots \\
	\end{pmatrix}
	$$
and
$$ \rowSums(F) = \bordermatrix{
	~ & ~ \cr
	1 & 9 \cr
	3 & 8 \cr
	4 & 8 \cr
	6 & 6 \cr
	7 & 6 \cr
	}
	$$
Which gives the expexted number of steps starting from each state
to first hit $9$.
Thus the answer is $9$.

\subsection{(b)}
The modified transition matrix is
\begin{equation}
	P \coloneqq \bordermatrix{~ & 1 & 4 & 6 & 7 & 3 & 9 \cr
		1 & 2/4 & 0 & 0 & 1/4 & 1/4 & 0 \cr
		4 & 0 & 1/4 & 1/4 & 1/4 & 1/4 & 0 \cr
		6 & 0 & 1/4 & 1/4 & 1/4 & 0 & 1/4 \cr
		7 & 0 & 1/4 & 0 & 2/4 & 0 & 1/4 \cr
		3 & 0 & 0 & 0 & 0 & 1 & 0 \cr
		9 & 0 & 0 & 0 & 0 & 0 & 1 \cr
		}
\end{equation}
Solving for $F_m$ gives
$$ F_m \coloneqq (I - Q)^{-1} = \begin{pmatrix}
		\cdots \\
	\end{pmatrix} =
	  \bordermatrix{~ & 3 & 9 \cr
		1 & 5/8 & 3/8 \cr
		4 & 1/2 & 1/2 \cr
		6 & 1/4 & 3/4 \cr
		7 & 1/4 & 3/4 \cr
		}
$$
From this matrix we can see that the probability of landing on square 3 starting on square 6 is equal to 1/4.

\appendix
\section{Appendix, R code}
\lstinputlisting[language=R]{ass1.R}

\end{document}
