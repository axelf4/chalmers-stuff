\documentclass{article}
\usepackage[T1]{fontenc}
\usepackage[utf8]{inputenc}
\usepackage[swedish]{babel}
\usepackage{amsmath, amssymb, mathtools}
\usepackage{enumitem}
\usepackage{siunitx}
\usepackage{listings}
\usepackage{comment}

\DeclareMathOperator\Poisson{Pois}
\DeclareMathOperator\GammaDist{Gamma}
\DeclareMathOperator\Normal{Normal}
\DeclareMathOperator\Uniform{Uniform}

\title{Second Assignment in MVE550}
\author{Axel Forsman, Jonas Lauri}

\begin{document}
\maketitle

\section{Question 1}
\begin{enumerate}[label=(\alph*)]
\item Given an improper prior, $\pi(\lambda) \propto_{\lambda} 1/\lambda$, and the first five steps of the branching process, compute the posterior.
\item Numerically compute the probability of extinction given a fixed $\lambda$ and that the process has reached size 7.
\item Compute the probability of extinction but take $\lambda$:s uncertainty into account by integrating out the uncertainity. 
\item Draw multiple samples of $\lambda$ to account for the uncertainty in the branching process. Compare with (d).
\item Do a maximum likelihood estimate for $\lambda$. What is the probability for extinction with this $\hat\lambda$?
\end{enumerate}

\subsection{(a)}
Let $Z_n \coloneqq \sum_{i=1}^{Z_{n-1}} X_i$,
where $Z_n$ is the size of each generation,
$X \sim \Poisson(\lambda)$, and $X, X_1, X_2, \ldots$ are i.i.d.
Then since the offspring distribution is a Poisson distribution
$$ G_X(s) = \exp(\lambda (s - 1)), \quad
G_{Z_n}(s) = \prod_{i=1}^{Z_{n-1}} G_X(s) = \exp(Z_{n-1} \lambda (s - 1)) $$

Using $ \mathbb P(Z_n = k) = G_{Z_n}^{(k)}(0) / k! $
we get for the different generations
\begin{align*}
	\mathbb P(Z_1 = 1) &= \lambda e^{-\lambda} &\quad
	\mathbb P(Z_2 = 2) &= \frac12 \lambda^2 e^{-\lambda} \\
	\mathbb P(Z_3 = 3) &= \frac16 (2\lambda)^2 e^{-\lambda} &\quad
	\mathbb P(Z_4 = 7) &= \frac1{7!} (3\lambda)^6 e^{-\lambda}
\end{align*}
where $\mathbb P(Z_0 = 1) = 1$ is known.
Putting everything together,
using the fact that everything is independent we get for the likelihood
$$ \pi(\text{data} \mid \lambda) = \frac{3^5}{7!} \lambda^{11} e^{-4\lambda}
	\propto_\lambda \GammaDist(12, 4) $$
and the posterior becomes
$$ \pi(\lambda \mid \text{data}) \propto_\lambda \lambda^{10} e^{-4\lambda}
	\propto_\lambda \GammaDist(11, 4)
	\implies \pi(\lambda \mid \text{data}) = \frac{\lambda^{10} e^{-4\lambda}}{\int_\Lambda \lambda^10 e^{-4\lambda} \, d\lambda} $$

\subsection{(c)}
Using that $\pi(\text{extinction} \mid \lambda, \text{data})$ is known
and given by the procedure derived in (b) we get
\begin{align*}
\pi(\text{extinction} \mid \text{data}) &= \int_{\Lambda \coloneqq \mathbb R^+} \pi(\text{extinction}, \lambda \mid \text{data}) \, d\lambda \\
	&= \int_\Lambda \pi(\text{extinction} \mid \lambda, \text{data}) \, \pi(\lambda \mid \text{data}) \, d\lambda
\end{align*}

\subsection{(d)}
To account for the uncertainty in $\lambda$ we simulate for
a large number of $\lambda$:s drawn from the posterior distribution
the extinction probability.
% TODO Add result

\subsection{(e)}
Finding the maximum of $\pi(\text{data} \mid \lambda)$ with respect to $\lambda$
yield the maximum likelihood estimate
$$ \hat\lambda = \ldots $$ % TODO
for which the simulated probability of extinction becomes

\section{Question 2}
\begin{enumerate}[label=(\alph*)]
\item Plot the data. 
\item Given a prior, $$\pi(\theta) = \Normal (\theta_1; 10, 10^2) \GammaDist (\theta_2; 1/2, 1/10) \Uniform (\theta_3; 0, 1) \GammaDist (\theta_4; 2, 2)$$ 
and a model for the data so that the likelihood becomes $$\pi (x_i \mid y_i) = \Normal(\theta_1 + \theta_2 \sin(x_i - \theta_3) 2\pi, \theta_4^2)$$
compute the \textit{logarithm} of the ratio $$\frac{\pi(\theta' \mid \text{data})}{\pi(\theta \mid \text{data})}$$
by computing the logarithms of the terms with a given $\theta$ and $\theta'$.
\item Use the Metropolis Hastings algorithm for 10 000 steps. \\
The proposal function should add $\Normal(0, 0.1)$ to $\theta_1$. 
To $\theta_2$ and $\theta_4$ it should add $\Normal (0, 0.1)$ and then take the absolute value. 
To $\theta_3$ we add $\Uniform(-0.1, 0.1)$ and make sure the result is between 0 and 1. 
\item Repeat the above process 100 times and report the means and standard deviations of $\theta$. 
\item Using the simultated values, answer the following question: \\
Taking into account the uncertainty in the parameters, what is the probability that a new observation with $x = 0.6$ will result in a response $y$ that is greater than 7.4?
\end{enumerate}

\subsection{(a)}
It is generally a good idea to
compute the logarithm of the partaking factors individually
since it can lead to better numberical stability,
since you'll add small numbers instead of multiplying them.

\appendix
\section{Appendix, R code}
% \lstinputlisting[language=R]{ass2.R}

\end{document}
