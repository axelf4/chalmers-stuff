\documentclass{article}
\usepackage[T1]{fontenc}
\usepackage[utf8]{inputenc}
\usepackage[swedish]{babel}
\usepackage{amsmath, amssymb, mathtools}
\usepackage{enumitem}
\usepackage{siunitx}
\usepackage{listings}
\usepackage{comment}
\usepackage{tikz}
\usepackage{afterpage}
\usepackage{booktabs}

\makeatletter
\def\fps@figure{hbtp}
\def\fps@table{hbtp}
\makeatother

\sisetup{
	round-mode      = places,
	round-precision = 3
	}


\lstset{
	breaklines=true,
	postbreak=\mbox{\textcolor{red}{$\hookrightarrow$}\space},
	}

\DeclareMathOperator\Poisson{Pois}
\DeclareMathOperator\GammaDist{Gamma}
\DeclareMathOperator\Normal{Normal}
\DeclareMathOperator\Uniform{Uniform}

\title{Second Assignment in MVE550}
\author{Axel Forsman, Jonas Lauri}

\begin{document}
\maketitle

\section{Question 1}
\begin{enumerate}[label=(\alph*)]
\item Given an improper prior, $\pi(\lambda) \propto_{\lambda} 1/\lambda$, and the first five steps of the branching process, compute the posterior.
\item Numerically compute the probability of extinction given a fixed $\lambda$ and that the process has reached size 7.
\item Compute the probability of extinction but take $\lambda$:s uncertainty into account by integrating out the uncertainity.
\item Draw multiple samples of $\lambda$ to account for the uncertainty in the branching process. Compare with (d).
\item Do a maximum likelihood estimate for $\lambda$. What is the probability for extinction with this $\hat\lambda$?
\end{enumerate}

\subsection{(a)}
Let $Z_n \coloneqq \sum_{i=1}^{Z_{n-1}} X_i$,
where $Z_n$ is the size of each generation,
$X \sim \Poisson(\lambda)$, and $X, X_1, X_2, \ldots$ are i.i.d.
Then since the offspring distribution is a Poisson distribution
$$ G_X(s) = \exp(\lambda (s - 1)), \quad
G_{Z_n}(s) = \prod_{i=1}^{Z_{n-1}} G_X(s) = \exp(Z_{n-1} \lambda (s - 1)) $$

Using $ \mathbb P(Z_n = k) = G_{Z_n}^{(k)}(0) / k! $
we get for the different generations
\begin{align*}
	\mathbb P(Z_1 = 1) &= \lambda e^{-\lambda} &\quad
	\mathbb P(Z_2 = 2) &= \frac12 \lambda^2 e^{-\lambda} \\
	\mathbb P(Z_3 = 3) &= \frac16 (2\lambda)^2 e^{-\lambda} &\quad
	\mathbb P(Z_4 = 7) &= \frac1{7!} (3\lambda)^6 e^{-\lambda}
\end{align*}
where $\mathbb P(Z_0 = 1) = 1$ is known.
Putting everything together,
using the fact that everything is independent we get for the likelihood
$$ \pi(\text{data} \mid \lambda) = \frac{3^5}{7!} \lambda^{11} e^{-4\lambda}
	\propto_\lambda \GammaDist(12, 4) $$
and the posterior becomes
$$ \pi(\lambda \mid \text{data}) \propto_\lambda \lambda^{10} e^{-4\lambda}
	\propto_\lambda \GammaDist(11, 4) $$

\subsection{(c)}
Using that $\pi(\text{extinction} \mid \lambda, \text{data})$ is known
and given by the procedure derived in (b) we get
\begin{align*}
	\pi(\text{extinction} \mid \text{data}) &= \int_{\Lambda \coloneqq \mathbb R^+} \pi(\text{extinction}, \lambda \mid \text{data}) \, d\lambda \\
		&= \int_\Lambda \pi(\text{extinction} \mid \lambda, \text{data}) \, \pi(\lambda \mid \text{data}) \, d\lambda
		\approx 0.004682
\end{align*}

\subsection{(d)}
To account for the uncertainty in $\lambda$ we simulate for
a large number of $\lambda$:s drawn from the posterior distribution
the extinction probability.
The result becomes $0.004845$.
We see that it is approximately what we saw in (c).

\subsection{(e)}
Finding the maximum of $\pi(\text{data} \mid \lambda)$ with respect to $\lambda$
yields the maximum likelihood estimate
$$ \hat\lambda_\text{mle} = \num{2.74998059503805} $$
for which the simulated probability of extinction becomes
$$ \texttt{extinctionProbability}(\hat\lambda_\text{mle}) \approx 0 $$
The simulation is unable to pick up any process that went extinct,
even though we know from theory that the probability is non-zero.

\section{Question 2}
\begin{enumerate}[label=(\alph*)]
\item Plot the data.
\item Given a prior, $$\pi(\theta) = \Normal (\theta_1; 10, 10^2) \GammaDist (\theta_2; 1/2, 1/10) \Uniform (\theta_3; 0, 1) \GammaDist (\theta_4; 2, 2)$$
and a model for the data so that the likelihood becomes $$\pi (x_i \mid y_i) = \Normal(\theta_1 + \theta_2 \sin(x_i - \theta_3) 2\pi, \theta_4^2)$$
compute the \textit{logarithm} of the ratio $$\frac{\pi(\theta' \mid \text{data})}{\pi(\theta \mid \text{data})}$$
by computing the logarithms of the terms with a given $\theta$ and $\theta'$.
\item Use the Metropolis Hastings algorithm for 10 000 steps. \\
The proposal function should add $\Normal(0, 0.1)$ to $\theta_1$.
To $\theta_2$ and $\theta_4$ it should add $\Normal (0, 0.1)$ and then take the absolute value.
To $\theta_3$ we add $\Uniform(-0.1, 0.1)$ and make sure the result is between 0 and 1.
\item Repeat the above process 100 times and report the means and standard deviations of $\theta$.
\item Using the simultated values, answer the following question: \\
Taking into account the uncertainty in the parameters, what is the probability that a new observation with $x = 0.6$ will result in a response $y$ that is greater than 7.4?
\end{enumerate}

\subsection{(a)}
The data is plotted in figure~\ref{fig:data_plot}.

\begin{figure}
	\centering
	% Created by tikzDevice version 0.12.3 on 2019-12-05 15:24:30
% !TEX encoding = UTF-8 Unicode
\begin{tikzpicture}[x=1pt,y=1pt]
\definecolor{fillColor}{RGB}{255,255,255}
\path[use as bounding box,fill=fillColor,fill opacity=0.00] (0,0) rectangle (216.81,216.81);
\begin{scope}
\path[clip] ( 49.20, 61.20) rectangle (191.61,167.61);
\definecolor{drawColor}{RGB}{0,0,0}

\path[draw=drawColor,line width= 0.4pt,line join=round,line cap=round] ( 54.47, 84.93) circle (  2.25);

\path[draw=drawColor,line width= 0.4pt,line join=round,line cap=round] ( 58.82, 88.34) circle (  2.25);

\path[draw=drawColor,line width= 0.4pt,line join=round,line cap=round] ( 61.72, 86.85) circle (  2.25);

\path[draw=drawColor,line width= 0.4pt,line join=round,line cap=round] ( 61.72, 73.44) circle (  2.25);

\path[draw=drawColor,line width= 0.4pt,line join=round,line cap=round] ( 64.62, 65.14) circle (  2.25);

\path[draw=drawColor,line width= 0.4pt,line join=round,line cap=round] ( 74.76, 71.31) circle (  2.25);

\path[draw=drawColor,line width= 0.4pt,line join=round,line cap=round] ( 82.01, 68.33) circle (  2.25);

\path[draw=drawColor,line width= 0.4pt,line join=round,line cap=round] ( 89.25, 86.63) circle (  2.25);

\path[draw=drawColor,line width= 0.4pt,line join=round,line cap=round] ( 90.70, 80.25) circle (  2.25);

\path[draw=drawColor,line width= 0.4pt,line join=round,line cap=round] (102.29, 74.93) circle (  2.25);

\path[draw=drawColor,line width= 0.4pt,line join=round,line cap=round] (106.64, 97.06) circle (  2.25);

\path[draw=drawColor,line width= 0.4pt,line join=round,line cap=round] (106.64,102.17) circle (  2.25);

\path[draw=drawColor,line width= 0.4pt,line join=round,line cap=round] (108.09,107.28) circle (  2.25);

\path[draw=drawColor,line width= 0.4pt,line join=round,line cap=round] (116.78,120.68) circle (  2.25);

\path[draw=drawColor,line width= 0.4pt,line join=round,line cap=round] (121.13,122.60) circle (  2.25);

\path[draw=drawColor,line width= 0.4pt,line join=round,line cap=round] (122.58,138.13) circle (  2.25);

\path[draw=drawColor,line width= 0.4pt,line join=round,line cap=round] (148.66,163.67) circle (  2.25);

\path[draw=drawColor,line width= 0.4pt,line join=round,line cap=round] (163.15,158.14) circle (  2.25);

\path[draw=drawColor,line width= 0.4pt,line join=round,line cap=round] (164.60,158.35) circle (  2.25);

\path[draw=drawColor,line width= 0.4pt,line join=round,line cap=round] (171.85,134.94) circle (  2.25);

\path[draw=drawColor,line width= 0.4pt,line join=round,line cap=round] (179.09,112.38) circle (  2.25);

\path[draw=drawColor,line width= 0.4pt,line join=round,line cap=round] (184.89,113.66) circle (  2.25);

\path[draw=drawColor,line width= 0.4pt,line join=round,line cap=round] (186.34,114.94) circle (  2.25);
\end{scope}
\begin{scope}
\path[clip] (  0.00,  0.00) rectangle (216.81,216.81);
\definecolor{drawColor}{RGB}{0,0,0}

\path[draw=drawColor,line width= 0.4pt,line join=round,line cap=round] ( 50.13, 61.20) -- (166.05, 61.20);

\path[draw=drawColor,line width= 0.4pt,line join=round,line cap=round] ( 50.13, 61.20) -- ( 50.13, 55.20);

\path[draw=drawColor,line width= 0.4pt,line join=round,line cap=round] ( 79.11, 61.20) -- ( 79.11, 55.20);

\path[draw=drawColor,line width= 0.4pt,line join=round,line cap=round] (108.09, 61.20) -- (108.09, 55.20);

\path[draw=drawColor,line width= 0.4pt,line join=round,line cap=round] (137.07, 61.20) -- (137.07, 55.20);

\path[draw=drawColor,line width= 0.4pt,line join=round,line cap=round] (166.05, 61.20) -- (166.05, 55.20);

\node[text=drawColor,anchor=base,inner sep=0pt, outer sep=0pt, scale=  1.00] at ( 50.13, 39.60) {0.0};

\node[text=drawColor,anchor=base,inner sep=0pt, outer sep=0pt, scale=  1.00] at ( 79.11, 39.60) {0.2};

\node[text=drawColor,anchor=base,inner sep=0pt, outer sep=0pt, scale=  1.00] at (108.09, 39.60) {0.4};

\node[text=drawColor,anchor=base,inner sep=0pt, outer sep=0pt, scale=  1.00] at (137.07, 39.60) {0.6};

\node[text=drawColor,anchor=base,inner sep=0pt, outer sep=0pt, scale=  1.00] at (166.05, 39.60) {0.8};

\path[draw=drawColor,line width= 0.4pt,line join=round,line cap=round] ( 49.20, 71.53) -- ( 49.20,156.65);

\path[draw=drawColor,line width= 0.4pt,line join=round,line cap=round] ( 49.20, 71.53) -- ( 43.20, 71.53);

\path[draw=drawColor,line width= 0.4pt,line join=round,line cap=round] ( 49.20, 92.81) -- ( 43.20, 92.81);

\path[draw=drawColor,line width= 0.4pt,line join=round,line cap=round] ( 49.20,114.09) -- ( 43.20,114.09);

\path[draw=drawColor,line width= 0.4pt,line join=round,line cap=round] ( 49.20,135.37) -- ( 43.20,135.37);

\path[draw=drawColor,line width= 0.4pt,line join=round,line cap=round] ( 49.20,156.65) -- ( 43.20,156.65);

\node[text=drawColor,rotate= 90.00,anchor=base,inner sep=0pt, outer sep=0pt, scale=  1.00] at ( 34.80, 71.53) {3};

\node[text=drawColor,rotate= 90.00,anchor=base,inner sep=0pt, outer sep=0pt, scale=  1.00] at ( 34.80, 92.81) {4};

\node[text=drawColor,rotate= 90.00,anchor=base,inner sep=0pt, outer sep=0pt, scale=  1.00] at ( 34.80,114.09) {5};

\node[text=drawColor,rotate= 90.00,anchor=base,inner sep=0pt, outer sep=0pt, scale=  1.00] at ( 34.80,135.37) {6};

\node[text=drawColor,rotate= 90.00,anchor=base,inner sep=0pt, outer sep=0pt, scale=  1.00] at ( 34.80,156.65) {7};

\path[draw=drawColor,line width= 0.4pt,line join=round,line cap=round] ( 49.20, 61.20) --
	(191.61, 61.20) --
	(191.61,167.61) --
	( 49.20,167.61) --
	( 49.20, 61.20);
\end{scope}
\begin{scope}
\path[clip] (  0.00,  0.00) rectangle (216.81,216.81);
\definecolor{drawColor}{RGB}{0,0,0}

\node[text=drawColor,anchor=base,inner sep=0pt, outer sep=0pt, scale=  1.00] at (120.41, 15.60) {x};

\node[text=drawColor,rotate= 90.00,anchor=base,inner sep=0pt, outer sep=0pt, scale=  1.00] at ( 10.80,114.41) {y};
\end{scope}
\end{tikzpicture}

	\caption{Plot of the data from the file \texttt{Regressiondata.txt}. \label{fig:data_plot}}
\end{figure}
% \afterpage{\clearpage}

\subsection{(b)}
It is generally a good idea to
compute the logarithm of the partaking factors individually
since it can lead to better numberical stability,
since you'll add small numbers instead of multiplying them.

\subsection{(c)}
The resulting $\theta$ is
$(\num{1327.19003751346}, \num{90.7653725828567}, \num{0.0229769401252709}, \num{651.369683888707})$.

\subsection{(d)}
The means and standard deviations of the simulated $\theta$ components
are shown in table~\ref{tab:theta_means_sds}.

\begin{table}
	\caption{Mean and standard deviations for the different components of $\theta$,
	that were simulated 100 times as in (c). \label{tab:theta_means_sds}}
	\centering
	\begin{tabular}{l r r r r} \toprule
		~ & \multicolumn{1}{c}{$\theta_1$} & \multicolumn{1}{c}{$\theta_2$} & \multicolumn{1}{c}{$\theta_3$} & \multicolumn{1}{c}{$\theta_4$} \\
		\midrule
		\textbf{Mean:} &
		\num{51.3363353789917} &
		\num{58.4570863747811} &
		\num{0.539136016514659} &
		\num{806.135765325465} \\
		\textbf{SD:} &
		\num{692.104989310555} &
		\num{25.1757118926179} &
		\num{0.31481735663175} &
		\num{44.7867130184426} \\
		\bottomrule
	\end{tabular}
\end{table}
% \afterpage{\clearpage}

\subsection{(e)}
Since $y \mid x, \theta$ is normally distributed,
we can take the average of the result of \texttt{1 - pnorm(7.4, ...)}
with the correct parameters
across the different simulated $\theta$:s,
and in doing so account for uncertainty in $\theta$.
The result comes out to be
$$ \mathbb P(y > 7.4 \mid x = 0.6) \approx \num{0.5123227} $$

\appendix
\section{Appendix, R code}
\lstinputlisting[language=R]{ass2.R}

\end{document}
