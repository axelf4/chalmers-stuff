\documentclass{article}
\usepackage[T1]{fontenc}
\usepackage[utf8x]{inputenc}
\usepackage[english]{babel}
\usepackage{amsmath, amsfonts, mathtools}
\usepackage{booktabs}
\usepackage{siunitx}
\usepackage{eurosym}
\usepackage{listings}
\usepackage{lmodern, microtype}

\sisetup{
	round-mode = figures,
	round-precision = 3,
}
\DeclareSIUnit\hectare{ha}
\lstset{
	breaklines=true,
	postbreak=\mbox{$\hookrightarrow$\space},
	breakindent = 5pt,
	breakautoindent = false,
}

\title{Biodiesel supply chain}
\author{Axel Forsman, Jonas Lauri}

\begin{document}
\maketitle
Collectively approximately 15 hours were spent on this assignment.

\section{Problem 1}

First the given values were converted into SI units,
see table~\ref{tab:crop_values},
and the effective costs of the respective products, taking into account tax,
were calculated, see table~\ref{tab:product_costs}.

\begin{table}
	\centering
	\caption{Crop values converted to SI units. \label{tab:crop_values}}
	\begin{tabular}{l r r r}
		\toprule
		Crop & Yield [kg/ha] & Water demand [l/ha] & Oil content [l/kg] \\
		\midrule
		Soybeans & \num{2.6e3} & \num{5.0e6} & 0.178 \\
		Sunflower seeds & \num{1.4e3} & \num{4.2e6} & 0.216 \\
		Cotton seeds & \num{0.9e3} & \num{1.0e6} & 0.433 \\
		\bottomrule
	\end{tabular}
\end{table}

\begin{table}
	\centering
	\caption{Effective product costs. \label{tab:product_costs}}
	\begin{tabular}{l r}
		\toprule
		Product & Effective price [\euro/l] ($\text{price} \cdot (1 - \text{tax})$) \\
		\midrule
		B5 & 1.144 \\
		B30 & 1.2255 \\
		B100 & 1.16 \\
		\bottomrule
	\end{tabular}
\end{table}

Let $C = \{\text{soy}, \text{sunflower}, \text{cotton}\}$ be
the set of crops and
$P = \{\text{B5}, \text{B30}, \text{B100}\}$ be the set of products.
Define functions
$$ \text{yield}, \text{waterdemand}, \text{oilcontent} : C \to \mathbb R $$
for the different types of crops,
and for the three products
$$ \text{biodieselprop}, \text{effectiveprice} : P \to \mathbb R $$
where $\text{effectiveprice}$ gives
$\text{price} * (1 - \text{tax})$ for the product in question.

Let $x_c \ge 0$ be the area in hectar to grow of crop $c \in C$,
and let $y_p \ge 0$ be number of litres to produce of product $p \in P$,
see figure~\ref{fig:productflow}.
The number of litres of vegetable oil produced then becomes
$$ \text{vegoil} = \sum_{c \in C} \text{oilcontent}(c) \, \text{yield}(c) \, x_c $$
with the water and area limits
$$ \sum_c \text{waterdemand}(c) \, x_c \le \SI{5000}{\mega\liter} \quad
\sum_c x_c \le \SI{1600}{\hectare} $$
Since \emph{all} oil is transesterificated the total amount of biodiesel is
$$ \text{biodieselamount} = 0.9 \text{vegoil} $$
The amount of petrol used is
$$ \text{petrolusage} = \sum_{p \in P} (1 - \text{biodieselprop}(p)) \, y_p $$
The petrol and delivery constraints become
$$ \text{petrolusage} \le \SI{150000}{\liter} \quad \sum y \ge \SI{280000}{\liter} $$
We cannot use more biodiesel than we have produced
$$ \sum_p \text{biodieselprop}(p) \, y_p \le \text{biodieselamount} $$
which relates the crop and product sides.
The objective  is then to maximize the profits,
i.e. the incomes minus the price of methanol and petrol,
$$ z = \sum_p \text{effectiveprice}(p) \, y_p - \num{1.5} \cdot 0.2 \, \text{vegoil} - 1 \, \text{petrolusage} $$

\begin{figure}
	\centering
	\def\svgwidth{\textwidth}
	\input{drawing.pdf_tex}
	\caption{The product flow of the problem
	showing the optimization variables. \label{fig:productflow}}
\end{figure}

\section{Problem 2}
The JuMP implementation of the model is given in appendix~\ref{app:code}.

The optimal value of the objective was found to be
\num{5.481630343e+05} \euro.
The values of the variables at the solution is shown in table~\ref{tab:optimal_values},
and we see that they look plausible by virtue of the usage of `free' biodiesel
from the crops seeming to have been maximized.

\begin{table}
	\centering
	\caption{Values of variables at optimum for the original solution. \label{tab:optimal_values}}
	\begin{tabular}{l r}
		\toprule
		Crop       & Value at optimum               \\
		\midrule
		Soybeans   & \SI{850}{\hectare}             \\
		Sunflowers & \SI{0}{\hectare}               \\
		Cotton     & \SI{750}{\hectare}             \\
		\bottomrule
		\toprule
		Product    & Value at optimum               \\
		\midrule
		B5         & \SI{0}{\liter}                 \\
		B30        & \SI{214285.7142857143}{\liter} \\
		B100       & \SI{552803.7857142857}{\liter} \\
		\bottomrule
	\end{tabular}
\end{table}

\section{Problem 3}
\subsection{(a) Availability of petrol, water and area}
% TODO I think the question means feasible as in there exists a solution
Using \verb+lp_rhs_perturbation_range(petrol_constraint)+ we find that
the current basis is always optimal regardless of the petrol limit,
that is the model will stay feasible if we only change
the petrol limit in a vacuum.
Otherwise the delivery constraint will limit what models are feasible.
With less biodiesel available,
we will be forced to use the more expensive petrol,
possible falling back to making B5.
The water and area constraints are fairly similar;
the LHS:s only differ on the coefficients for $x$.
Still how they matter will depend on what happens to be
the best crop distribution.

\subsection{(b) Potential gain from increasing the availabilities}
Any increase in availabilities can only increase our potential gains.
Increasing petrol will increase B30, and after it is maxed out then make B5.
With unlimited petrol, one would only make B5 at a potential profit of 1.234179e7.
Hence, B5 is most profit per litre.
B30 is the most profit per litre of petrolium.
Increasing water supply will make us able to grow more soy and increase yield.
This will increase the amount of B100.
Increasing the availiable area will make us able to grow more cotton and increase yield.
This will increase the amount of B100.
This increased yield will not affect B30 production as we already max it out.

In the dual each constraint is represented by one variable.
Their optimal values are the \emph{shadow prices} -
how much profit increase would correspond to upping the resource amount
in the constraint RHS by one unit:
Decreasing petrol availability by \num{10000} gives a reduction in profits by \num{2535.7} \euro. 
This is what one would expect from table ~\ref{tab:shadowprices}.
This relation can be checked for area and water aswell. 

\begin{table}
	\centering
	\caption{Shadow prices for the dual variables. \label{tab:shadowprices}}
	\begin{tabular}{l r}
		\toprule
		Constraint & Value per one unit \\
		\midrule
		Petrol & 0.2535714285714 \\
		Area & 276.3402 \\
		Water & 1.3596599999999994e-5 \\
		\bottomrule
	\end{tabular}
\end{table}

\subsection{(c) Sunflower}
Even if the model is augmented such that
the water demand of sunflower is zero,
the optimal solution still does not use any sunflower.
One could bring up the water demand of both soy and cotton
to make growing anything other than sunflower unfeasible.
Otherwise, if still only changing one parameter in isolation,
either yield or oil content has to be changed -
which one will not matter since they are only ever used multiplied together.
Using binary search on
$\text{yield of sunflower} \in [\SI{1.40e3}{\kilo\gram\per\hectare}, 3000] \setminus 10^{-3} \times (0, 1)$,
for a solution where sunflower is grown,
we get that the yield has to be increased to around \SI[round-precision=6]{2074.907445907593}{\kilo\gram\per\hectare}
for the growing of any sunflower to be profitable.

\subsection{(d) The price of petrol at 1.20 \euro/l.}\label{sec:3d}
The optimal value of the objective was found to be
\num{5.181630343e+05} \euro. 
Approximatly 3 000 \euro less than before, with petrol at 1 \euro/l.
The results are in table~\ref{tab:new_values}.

\begin{table}
	\centering
	\caption{Values of variables at optimum for the new solution in section~\ref{sec:3d}. \label{tab:new_values}}
	\begin{tabular}{l r}
		\toprule
		Crop       & Value at optimum               \\
		\midrule
		Soybeans   & \SI{850}{\hectare}             \\
		Sunflowers & \SI{0}{\hectare}               \\
		Cotton     & \SI{750}{\hectare}             \\
		\bottomrule
		\toprule
		Product    & Value at optimum               \\
		\midrule
		B5         & \SI{0}{\liter}                 \\
		B30        & \SI{214285.7142857143}{\liter} \\
		B100       & \SI{552803.7857142857}{\liter} \\
		\bottomrule
	\end{tabular}
\end{table}

\subsection{(e) Changes in tax-policy}
Any increase in tax will make the objective value decrease or stay the same. 
Any decrease in tax will make the objective value increase or stay the same. 
To have a difference on which products we make, the most profitable product per litre must change. 
With the original data the optimal usage of vegetable oil is making B30, followed by B100, and then B5. 
For the tax differences to affect which product is being made it must disrupt this order. 
Three important breaking points are: 
B100 more profitable than B30 for tax at 10\%. 
B100 more profitable than B5 for tax at 19\%. 
B30 as profitable as B5: $\textrm{tax}_5 = (\textrm{tax}_{30} + 0.11)/1.11$.
We can get a different result by B5 being more profitable than B30 or B100 being more profitable than B30 and B5. 
B5 more profitable than B30: This causes the production to make B5 at maximal capacity.  
Then with any remaining capacity, B100 would be made. 
B100 more profitable than B5 and B30: In that case there would only be B100 produced. 

\subsection{(f) Uncertainty of weather}\label{sec:3f}
Changing the water demand for all crops by some factor $f$
is the same as instead changing the availabile water amount
by $1/f$ due to linearity.
Changing water availablility is effectively the same as moving a hyperplane
and we know that we will stay in the same basis and in turn vertex
while doing so for perturbations of size
\verb+JuMP.lp_rhs_perturbation_range(water_constraint) = (-3.4e9, 3.0e9)+.
Therefore the optimum will move linearly
and since the objective function is a linear function,
we can construct the objective as a function of water availablility
by only knowing two points
$$ \left\{ \begin{array}{r l}
	z(w) &= z_0 + \frac{z_1 - z_0}{w_1 - w_0} (w - w_0) \\
	(z_0, w_0) &= (\num{5.481630343e+05}, \num{5000e6}) \\
	(z_1, w_1) &= (\num{5.345664343e+05}, \num{4000e6})
\end{array} \right.
	$$
where $z$ is the optimal objective, $w$ is the water limit
and $(z_0, w_0)$, $(z_1, w_1)$ are two points.

Furthermore, since the water constraint is active
(as shown by the constraint right-hand side \num{5.0e9} being equal to
\verb+JuMP.value(water_constraint)+)
any tweaking of the coefficients in the LHS of the constraint
will necessitate a change in optimal solution.
That is, the water demand for soybeans is not allowed to change one bit.

\subsection{(g) Optimal value of the objective function}
Following the same reasoning as in section~\ref{sec:3f},
the values of \verb+lp_rhs_perturbation_range+ for
\verb+petrol_constraint+, \verb+water_constraint+ and \verb+area_constraint+
are all non-empty ranges,
wherefore the objective function will react linearly to small-enough perturbations.
By querying at these points we get that
$$ z(\Delta) = z_0 + k \Delta $$
where $\Delta = (dP_\text{max}, dW_\text{max}, dA_\text{max})^T$ is the perturbation
and
$$ k = \frac1{10^4} \left[ \begin{array}{l}
	\num{5.506987486e+05} - z_0 \\
	\num{5.481631703e+05} - z_0 \\
	\num{1.487719714e+06} - z_0 \\
\end{array}\right]$$

\subsection{(h) A more sustainable objective function}
An easy way would be just taxing the unecofriendly options more. 
This would cause ecofriendly options to be priorizied. 
Examples of data which could affect how high the tax should be:
Amount of CO2 equivalents being released per product, amount of eventual hazardous substances being released, etc. 
Additionally there would be value in looking at the crops being grown. 
Cotton, being the most water efficient crop, could be prioritized using some kind of extra tax aswell.
For this, one could tax water usage. 
Other factors to take into account are, for example: 
The amount of pesticide being released per crop, how much land the crop needs, in what state the used land ends up in, 
ie. to what level does the crop drain the nutrients of the ground?
All this data could then be used to make appropriate tax levels.  
This is largely how ecofriendly options are treated in the real world.  
Of course one could give subsidies aswell, instead of taxing, but that is essentially the other side of the same coin. 

\appendix
\section{Julia code}\label{app:code}
\subsection{biodiesel.jl}
\lstinputlisting{biodiesel.jl}
\subsection{dat.jl}
\lstinputlisting{dat.jl}
\subsection{mod.jl}
\lstinputlisting{mod.jl}

\end{document}