\documentclass{article}
\usepackage[T1]{fontenc}
\usepackage[utf8]{inputenc}
\usepackage[swedish]{babel}
\usepackage{amsmath, amsfonts, amsthm, mathtools}
\usepackage{cancel}

\newtheorem{theorem}{Theorem}
\newtheorem{definition}{Definition}
\newtheorem{lemma}[theorem]{Lemma}
\newtheorem{proposition}{Proposition}
\newtheorem{corollary}{Corollary}[theorem]

\DeclarePairedDelimiter\abs{\lvert}{\rvert}
\newcommand{\norm}[1]{\left\lVert#1\right\rVert}
\DeclareMathOperator{\Deg}{deg}

\makeatletter
\let\oldabs\abs
\def\abs{\@ifstar{\oldabs}{\oldabs*}}
% \let\oldnorm\norm
% \def\norm{\@ifstar{\oldnorm}{\oldnorm*}}
\makeatother

\title{MVE030: Fourier analysis}
\author{Axel Forsman}

\begin{document}
\maketitle

\section{Fourier series}

\begin{definition}{Periodicity}
	A function $f$ is \emph{periodic} with period $p$ if
	$$ f(x + p) = f(x) \quad \forall x \in \mathbb R $$
	and this is the smallest $p$ for which this holds.
\end{definition}

\begin{lemma}
	If $f$ is periodic with period $p$ then
	$\int_a^{a + p} f \, dx$
	is the same for all $a \in \mathbb R$.
\end{lemma}

\begin{proof}
	Let
	$$ q(a) \coloneqq \int_a^{a + p} f(x) \, dx = \int_0^{a + p} f \, dx - \int_0^a f \, dx $$
	Then the fundamental theorem of calculus gives $ q'(a) = f(a + p) - f(a) = 0 $.
\end{proof}

\begin{definition}{Hilbert space}
	A complete (every Cauchy sequence converges) normed vector space
	with a hermitian scalar product, that is

	$$ \langle f, g \rangle = \overline{\langle g, f \rangle} $$
\end{definition}

\begin{definition}{(Working definition of) $L^2([a, b])$}
	All functions $f : [a, b] \to \mathbb C$ such that $\norm f < \infty$.
	Hilbert space with scalar product
	$$ \langle f, g \rangle_{L^2} = \int\limits_a^b f(x) \overline{g(x)} \, dx $$
\end{definition}

\begin{proposition}{Standard estimate}
	Assume $\lvert f(x) \rvert \le M$ for (almost every) $x \in [a, b]$, then
	$$ \int_a^b \lvert f(x) \rvert^2 \, dx \le M^2 (b - a) $$
	thus $f \in L^2([a, b])$.
\end{proposition}
\begin{proof}
	$$ \int_a^b \lvert f(x) \rvert^2 \, dx \le \int_a^b M^2 \, dx = M^2 (b - a) \qedhere $$
\end{proof}

\begin{corollary}
	If $f$ is continuous on $[a, b]$ then $f \in L^2([a, b])$.
\end{corollary}

\begin{proposition}
	Let $H$ Hilbert space:
	$$ \forall u, v \in H: \norm{u + v}^2 = \norm{u}^2 + 2 \Re \langle u, v \rangle + \norm{v}^2 $$
\end{proposition}

\begin{proposition}{Cauchy-Schwarz inequality}
	For any Hilbert space, $H$, and $u, v \in H$,
	$$ \abs{\langle u, v \rangle} \le \norm{u} \norm{v} $$
\end{proposition}

\begin{theorem}{Bessel's inequality}
	Let $H$ Hilbert space, $\{e_n\} \subset H$ orthonormal sequence, then
	$$ \sum_n \lvert \langle x, e_n \rangle \rvert^2 \le \norm{x}^2 \quad \forall f \in H $$
\end{theorem}
\begin{corollary}
	$ \lim_{n \to \infty_-^+} c_n = 0 $
\end{corollary}

\begin{theorem}{Three equivalent conditions for an orthonormal set to be an ONB}
	\begin{enumerate}
		\item If $f \in H, \langle f, \phi_n \rangle = 0 \; \forall n \implies f = 0$
		\item If $f \in H, \sum_n \langle f, \phi_n \rangle \, \phi_n = f$
		\item If $f \in H, \norm{f}^2 = \sum_n \abs{\langle f, \phi_n \rangle}^2$
	\end{enumerate}
\end{theorem}
\begin{proof}
	Start by showing (I) implies (II):
	Let $g \coloneqq \sum_n \langle f, \phi_n \rangle \, \phi_n \in H$.
	% TODO Is this needed
	% $$ \norm{g}^2 \overset{\text{Pythagorean}}{\underset{\text{Theorem}}=}
	% = \sum_n \norm{\langle f, \phi_n \rangle \, \phi_n}^2 = \sum_n \abs{\langle f, \phi_n \rangle}^2
	% \overset{\text{Bessel's}}{\underset{\text{inequality}}\le} \norm{f}^2 $$
	Fix $n$, then
	$$ \langle f - g, \phi_n \rangle = \langle f, \phi_n \rangle - \underbrace{\langle g, \phi_n \rangle}_{=\sum_m \langle f, \phi_n \rangle \, \langle \phi_m, \phi_n \rangle = \langle f, \phi_n \rangle} = 0 $$
	Thus (I) implies that $f = g$.
	For (II) implies (III):
	$$ \norm{f}^2 \overset{\text{(II)}}= \norm{\sum_n \langle f, \phi_n \rangle \, \phi_n}^2 \overset{\text{Inf}}{\underset{\text{Pyth}}=} \sum_n \norm{\langle f, \phi_n \rangle \, \phi_n}^2 = \sum_n \abs{\langle f, \phi_n \rangle}^2 $$
	For (III) implies (I): Assume $\langle f, \phi_n \rangle = 0 \; \forall n$, then, by (III)
	$$ \norm{f}^2 = \sum_n \abs{\langle f, \phi_n \rangle}^2 = \sum_n 0 = 0 \overset{\text{def of \,} \norm{\cdot}}\implies f = 0 $$
\end{proof}

\begin{definition}{Piecewise $C^k$}
	A function is piecewise $C^k$ on interval $I$ if it is $k$ times differentiable
	on $I \setminus S, \, S \coloneqq \{s_j\}_{j=1}^n$
	and $\forall j \, \exists \lim_{x \to {s_j}_-^+} f^{(j)}(x)$ for $l = 0, 1, \ldots, k$.
\end{definition}

\begin{theorem}{Pointwise convergence of Fourier series}
	Assume $f$ piecewise $C^1$ on $[-\pi, \pi]$ and $2\pi$ periodic on $\mathbb R$.
	Then $ S_N(x) \coloneqq \sum_{n = -N}^N c_n e^{inx} $ satisfies
	$$ \lim_{N \to \infty} S_N(x) = \frac{f(x_-) + f(x_+)}2 \quad \forall x \in \mathbb R $$
	When $f$ continuous at $x$ then $f(x_+) = f(x_-) = f(x)$.
\end{theorem}
\begin{proof}
	\begin{enumerate}
		\item Fix $x \in \mathbb R$
		\item Use definition of $c_n$:
			$ \displaystyle S_N(x) = \sum_{\lvert n \rvert \le N} \frac1{2\pi} \int\limits_{-\pi}^\pi f(y) e^{in (x - y)} \, dy $
		\item Change variables: Let $t \coloneqq y - x$. Then $y = t + x$, $dy = dt$
			\begin{align*}
				S_N(x) &= \sum_{\abs n \le N} \frac1{2\pi} \int\limits_{-\pi - x}^{\pi - x} \overbrace{f(t + x) e^{-int}}^{2\pi \text{ periodic}} \, dt \\
				&= \sum_{\abs n \le N} \frac1{2\pi} \int\limits_{-\pi}^{\pi} f(t + x) e^{-int} \, dt \\
				&= \int\limits_{-\pi}^{\pi} f(t + x) \left( \frac1{2\pi} \sum_{\abs n \le N} e^{-int} \right) \, dt
			\end{align*}
		\item Two facts about N\textsuperscript{th} Dirichlet kernel, $D_N(t) \coloneqq \frac1{2\pi} \sum_{-N}^N e^{\pm int}$:

			\begin{enumerate}
				\item $$ e^{int} + e^{-int} = 2 \cos nt \implies D_N(t) = \frac1{2\pi} \left(1 + \sum_{n = 1}^N 2\cos nt\right) $$
					$D_N$ is even: $\int_{-\pi}^0 D_N \, dt = \int_0^{\pi} D_N \, dt = \frac12$
				\item
					\begin{align*}
						D_N(t) &= \frac{e^{-iNt}}{2\pi} \sum_{n = 0}^{2N} e^{int} = \frac{e^{-iNt}}{2\pi} \frac{1 - e^{i(2N + 1)t}}{1 - e^{it}} \\
						&= \frac{e^{-iNt} - e^{i(N + 1)t}}{2\pi (1 - e^{it})}
					\end{align*}
			\end{enumerate}

			Our goal is to prove $\lim_{N \to \infty} \abs{S_N(x) - \frac{f(x_-) + f(x_+)}2} = 0$ or equivalently
			$\abs{\int_{-\pi}^\pi f(t + x) D_N(t) \, dt - \frac{f(x_-) + f(x_+)}2} \to 0, N \to \infty$
		\item $\frac12 f(x_-) = \int_{-\pi}^0 f(x_-) D_N(t) \, dt, \,
			\frac12 f(x_+) = \int_0^\pi f(x_+) D_N(t) \, dt$

			\begin{multline*}
				\abs{\int_{-\pi}^\pi f(t + x) D_N(t) \, dt - \int_{-\pi}^0 f(x_-)D_N(t) \, dt - \int_0^\pi f(x_+)D_N(t) \, dt} \\
				= \abs{\int_{-\pi}^0 (f(t + x) - f(x_-)) D_N(t) \, dt + \int_0^\pi (f(t + x) - f(x_+)) D_N(t) \, dt}
			\end{multline*}

		\item Substitute the second fact about $D_N$:
			\begin{multline*}
				\lvert \int_{-\pi}^0 \frac{e^{-iNt} - e^{i(N + 1)t}}{2\pi (1 - e^{it})} (f(t + x) - f(x_-)) \, dt \\
				+ \int_0^\pi \frac{e^{-iNt} - e^{i(N + 1)t}}{2\pi (1 - e^{it})} (f(t + x) - f(x_+)) \, dt \rvert
			\end{multline*}

		\item Define
			$$ g(t) \coloneqq \begin{cases}
				\frac{f(t + x) - f(x_-)}{1 - e^{it}} & -\pi \le t < 0 \\
				\frac{f(t + x) - f(x_+)}{1 - e^{it}} & 0 \le t \le \pi
			\end{cases} $$
			$$ \lim_{t \to 0_-} \frac{f(t + x) - f(x_-)}{1 - e^{it}} = \lim_{t \to 0_-} \frac{f(t + x) - f(x_-) t}{t (1 - e^{it})}
			= \frac{f'(x_-)}{-i e^{i(0)}} = \frac{f'(x_-)}{-i} $$
			where the second equality is due to $f$ being piecewise $C^1$.
			PSS $\lim_{t \to 0_+} g(t) = \frac{f'(x_+)}{-i}$
			Thus $g$ is piecewise $C^1$ too. Therefore $g$ is \emph{bounded} on $[-\pi, \pi]$.

			$$ \abs{S_N(x) - \frac{f(x_-) + f(x_+)}2}
			= \abs{\int_{-\pi}^\pi \left(\frac{e^{-iNt}}{2\pi} g(t) + \frac{e^{i(N + 1)t}}{2\pi} g(t) \right) \, dt} $$

		\item Bessel's inequality: For $g$ its $N$\textsuperscript{th} Fourier coefficient is
			$$ \frac1{2\pi} \int_{-\pi}^\pi g(t) e^{-iNt} \, dt \to 0, N \to \infty $$
			The $-N-1$\textsuperscript{st} Fourier coefficient is
			$$ \frac1{2\pi} \int_{-\pi}^\pi g(t) e^{i(N + 1)t} \, dt \to 0, N \to \infty \qedhere $$
	\end{enumerate}
\end{proof}

\begin{definition}{Fourier series}
	The Fourier series for a $2\pi$ periodic function is
	$$ \sum_{n \in \mathbb Z} c_n e^{inx} $$
\end{definition}

Want to differentiate/integrate solutions to PDE:s.

\begin{theorem}{On differentiation of Fourier series}
	Assume $f$ piecewise $C^1$ on $(-\pi, \pi)$, $2\pi$ periodic and continuous on $\mathbb R$.
	Then the Fourier coeffs $a_n, b_n, c_n$ and those of $f'$; $a'_n, b'_n, c'_n$; satisfy
	\begin{align*}
		a'_n &= n b_n & b'_n &= -n a_n & c'_n &= in c_n
	\end{align*}
\end{theorem}
\begin{proof}
	Use the definition of $c'_n$ and relate to $c_n$ through integration by parts
	\begin{align*}
		c'_n &\coloneqq \frac1{2\pi} \int_{-\pi}^\pi f'(x) e^{-inx} \, dx \\
		&= \frac1{2\pi} \left( \left. f(x) e^{-inx} \right\rvert_{-\pi}^\pi - \int_{-\pi}^\pi f(x) (-in) e^{-inx} \, dx \right) = in c_n
	\end{align*}
	where we have used that $f(x) e^{-inx}$ is $2\pi$ periodic.
\end{proof}

\begin{theorem}
	Assume $f$ $2\pi$ periodic, $f$ is $C^{k-1}$, $f^{(k-1)}$ is piecewise $C^1$, f is piecewise $C^k$. Then
	$$ \sum \abs{n^k c_n}^2 < \infty $$
	If $\abs{c_n} \le c \abs n^{-k - \alpha} \; \forall n, \text{for some} c>0, \alpha > 1$ then $f \in C^k$.
	(Idea: The rate of convergence of Fourier Series $\longleftrightarrow$
	how smooth $f$ is)
\end{theorem}

\begin{proposition}
	On $[-\pi, \pi]$, $\phi_n(x) \coloneqq \frac{e^{inx}}{\sqrt{2\pi}}, \, n \in \mathbb Z$ are orthonormal.
\end{proposition}
\begin{proof}
	$$ \langle \phi_n, \phi_m \rangle = \int\limits_{-\pi}^\pi \frac{e^{inx}}{\sqrt{2\pi}} \overline{\frac{e^{imx}}{\sqrt{2\pi}}} \, dx
	= \frac1{2\pi} \begin{cases}
		2\pi & n = m \\
		\left. \frac{e^{xi(n-m)}}{i(n-m)} \right\rvert_{x=-\pi}^\pi = 0 & n \ne m
	\end{cases} \qedhere $$
\end{proof}

\begin{theorem}{Best approximation theorem}
	Let $\{\phi_n\}_{n \ge 1} \subset H \text{ = Hilbert space}$ ONS.
	Then $\forall \sum_{n \ge 1} c_n \phi_n \in H, \, c_n \ in \mathbb C$
	$$ \norm{f - \sum_{n \ge q} \langle f, \phi_n \rangle} \le \norm{f - \sum_{n \ge q} c_n \phi_n} $$
	with equality iff $c_n = \langle f, \phi_n \rangle \forall n \ge 1$.
\end{theorem}
\begin{proof}
	Let $\hat f_n \coloneqq \langle f, \phi_n \rangle$.
	These are the Fourier coeffs of $f$ wrt. $\{\phi_n\}$.
	$\sum_n \hat f_n \phi_n$ is the Fourier series of $f$ wrt. $\{\phi_n\}$.
	Let $g \coloneqq \sum_n \hat f_n \phi_n$, and $\varphi \coloneqq \sum_n c_n \phi_n$.

	Idea: $\norm{f - \varphi}^2 = \norm{f - g + g - \varphi}^2 = \norm{f - g}^2 + 2\Re \langle f - g, g - \varphi \rangle + \norm{g - \varphi}^2$

	Deal with $\langle f - g, g - \varphi \rangle$. Want to show it is 0.

	\begin{multline*}
		\langle f - g, g - \varphi \rangle = \langle f, g \rangle - \langle f, \varphi \rangle - \langle g, g \rangle + \langle g, \varphi \rangle \\
		= 0
	\end{multline*}

	Therefore $\norm{f - \varphi}^2 \ge \norm{f - g}^2$ and equal iff $\norm{g - \varphi}^2 = 0$.

	$$ \norm{g - \varphi}^2 = \norm{sum_n (\hat f_n - c_n) \phi_n}^2 \overset{\text{Pythagoras}}{=} \sum_n \norm{(\hat f_n - c_n) \phi_n}^2 = \sum_n \abs{\hat f_n - c_n}^2 $$
	This is 0 iff $\abs{\hat f_n - c_n}^2 = 0 \; \forall n \iff \hat f_n = c_n \; \forall n$.
\end{proof}

Sturm-Leoville problems: ODE:s that are physically important and/or come
from PDE:s and separating variables.

\begin{definition}{Regular Sturm-Leoville Problem}
	\begin{enumerate}
		\item A differential operator $L(f) \coloneqq (r f')' + pf$
			with $r, p$ real valued functions, $r, r' \text{ and } p$ are continuous.
			The problem is for $f(x), \, x \in [a, b], r(x) > 0$.
		\item Boundary conditions (``Self-adjoint'')
			$$ B_i(f) = \alpha_i f(a) + \alpha'_i f'(a) + \beta_i f(b) + \beta'_i f'(b) = 0 \quad for i = 1, 2 $$
			Such that if $f, g$ both satisfy the boundary conditions then
			$$ \left. r (\bar g f' - \bar g' f) \right\rvert_a^b = 0 $$
		\item Positive continuous weight function $w(x)$ on $[a, b]$
	\end{enumerate}

	The SLP is to find all solutions $f$ `eigenfunctions' and `eigenvalues' $\lambda$
	such that
	$$ L(f) + \lambda w f = 0, B_i(f) = 0, i = 1, 2 $$
\end{definition}

\begin{theorem}{Adult Spectral Theorem}
	For \emph{every} regular SLP there is an orthogonal basis of eigenfunctions
	for $L_w^2((a, b))$ with eigenvalues $\{\lambda_n\} \subset \mathbb R, \lim_{n \to \infty} \lambda_n = \infty$.
	$$ L_w^2(a, b) \coloneqq \left\{f: \int_a^b \abs{f(x)}^2 w(x) \, dx < \infty\right\} $$
	Scalar product is $\langle f, g \rangle_w = \int_a^b f(x) \overline{g(x)} w(x) \, dx$.
\end{theorem}

\begin{corollary}
	The functions $\{e^{inx}\}_{n \in \mathbb Z}$ are an orthogonal basis for $L^2(-\pi, \pi)$.
\end{corollary}
\begin{proof}
	Consider
	$$ f'' = -\lambda f, \quad x \in [-\pi, \pi], f \text{ is } 2\pi \text{ periodic} $$
	The only solutions are $e^{inx}, \, n \in \mathbb Z$.
	Need to show that it is a regular SLP: The boundary conditions are
	$$ \left\{ \begin{array}{lr}
		f(-\pi) - f(\pi) = 0, & \alpha_1 = 1, \alpha'_1 = 0, \beta_1 = -1, \beta'_1 = 0 \\
		f'(-\pi) - f'(\pi) = 0, & \alpha_2 = 0, \alpha'_2 = 1, \beta_2 = 0, \beta'_2 = -1
	\end{array} \right. $$
	Assume $f, g$ satisfy the BC, thus $\left. \bar g f' - \bar g' f \right\rvert_{-\pi}^\pi = 0$, by periodicity.
	Therefore it is a regular SLP and the theorem says the solutions form an OB.
\end{proof}

\section{Fourier transform}

\begin{definition}{Fourier transform}
	Let $f \in L^1(\mathbb R), \, \xi \in \mathbb R$.
	$$ \hat f(\xi) \coloneqq \int_{\mathbb R} e^{-ix\xi} f(x) \, dx \left(= \int_{\mathbb R} f(x) \overline{e^{ix\xi}} \, dx  \text{ kont. version av Fourier koeff.}\right) $$
\end{definition}

Om $f \in L^1(\mathbb R)$, $\abs{\hat f(\xi)} = \abs{\int_{\mathbb R} e^{-ix\xi} f(x) \, dx} \le \int_{\mathbb R} \abs{f(x)} \, dx = \norm f_{L^1(\mathbb R)}$.

\begin{definition}{Convolution (Faltning)}
	Om $\int_{\mathbb R} f(x - y) g(y) \, dy \in \mathbb R$, är faltningen
	$$ f * g(x) \coloneqq \int_{\mathbb R} f(x - y) g(y) \, dy $$
\end{definition}

\begin{proposition}
	Assume $f, g, h \in L^2(\mathbb R)$
	\begin{enumerate}
		\item $\abs{f * g(x)} \le \norm f_{L^2(\mathbb R)} \norm g_{L^2(\mathbb R)}$
		\item $f * (ag + bh) = af * g + bf * h \quad \forall a, b \in \mathbb C \quad \text{(linear)}$
		\item $f * g = g * f \quad \text{(commutative)}$
		\item $f * (g * h) = (f * g) * h \quad \text{(associative)}$
	\end{enumerate}
\end{proposition}

\begin{proposition}
	If $f \in C^1(\mathbb R) \cap L^2(\mathbb R), f' \in L^2(\mathbb R), g \in L^2(\mathbb R)$ then
	$$ f * g \in C^1(\mathbb R) \quad (f * g)' = f' * g $$
\end{proposition}
% TODO Proof mollification

\begin{theorem}{Properties of the Fourier transform}
	Assume everything is well-defined
	\begin{enumerate}
		\item $\mathcal F(f(x - a))(\xi) = e^{ix} \hat f(\xi)$
		\item $\mathcal F(f')(\xi) = i\xi \hat f(\xi)$
		\item $\mathcal F(x f(x))(\xi) = i \mathcal F(f')(\xi)$
		\item $\mathcal F(f * g)(\xi) = \hat f(\xi) \hat g(\xi)$
	\end{enumerate}
\end{theorem}

\begin{theorem}{Fourier inversion theorem (FIT)}
	(There is a well-defined extension of $\mathcal F$ to $L^2$)
	For any $f \in L^2(\mathcal R)$
	$$ f(x) = \frac1{2\pi} \int_{\mathbb R} e^{ix\xi} \hat f(\xi) \, d\xi $$
\end{theorem}

\begin{theorem}{Plancherel}
	For any $f, g \in L^2$ ($\implies \hat f, \hat g \in L^2$)
	$$ \langle \hat f, \hat g \rangle = 2\pi \langle f, g \rangle $$
	Thus $\norm{\hat f}_{L^2(\mathbb R)}^2 = 2\pi \norm f_{L^2(\mathbb R)}^2$
\end{theorem}
\begin{proof}
	\begin{align*}
		2\pi \langle f, g \rangle &= 2\pi \int_{\mathbb R} f(x) \overline{g(x)} \, dx
		\overset{FIT}= 2\pi \int_{\mathbb R} \frac1{2\pi} \int_{\mathbb R} e^{ix\xi} \hat f(\xi) \, d\xi \, \overline{g(x)} \, dx \\
								  &= \int_{\mathbb R} \hat f(\xi) \overline{\int_{\mathbb R} e^{-ix\xi} g(x) \, dx} \, d\xi
								  = \int_{\mathbb R} \hat f(\xi) \overline{\hat g(\xi)} \, d\xi = \langle \hat f, \hat g \rangle \qedhere
	\end{align*}
\end{proof}

\begin{lemma}{Riemann-Lebesgue}
	For any $f \in L^2(\mathbb R)$, $\lim_{\xi \to \pm \infty} \hat f(\xi) = 0$.
\end{lemma}

\begin{theorem}{Convolution approximation}
	Assume $g \in L^1(\mathbb R)$ with $\int_{\mathbb R} g \, dx = 1$.
	Define $\alpha = \int_{-\infty}^0 g \, dx$ and $\beta = \int_0^\infty g \, dx$.
	Assume $f$ is piecewise continuous on $\mathbb R$ and either
	(I) $f$ is bounded, or (II) $g$ vanishes outside a bounded interval.
	Let $g_\epsilon(x) = \frac{g(x / \epsilon)}\epsilon$.
	Then
	$$ \lim_{\epsilon \downarrow 0} f * g_\epsilon(x) = \alpha f(x_+) + \beta f(x_-) \quad \forall x \in \mathbb R $$
	If $f$ is continuous then $\lim_{\epsilon \downarrow 0} f * g(x) = f(x)$.
\end{theorem}
\begin{proof}
	Idea: Make into `left' and `right' sides:
	\begin{multline*}
		\overbrace{\int_{-\infty}^0 f(x - y) g_\epsilon(y) \, dy - \int_{-\infty}^0 g(y) f(x_+) \, dy}^{L_\epsilon} \\
		+ \underbrace{\int_0^\infty f(x - y) g(y) \, dy - \int_0^\infty g(y) f(x_-) \, dy}_{R_\epsilon} \overset?\to 0, \epsilon \to 0
	\end{multline*}
	The proofs for the separate sides are analogous, so pick left.
	Fix $\delta > 0$.
	By change of variables, $\int_{-\infty}^0 g(y) \, dy = \int_{-\infty}^0 \frac{g(z / \epsilon)}\epsilon \, dz$.
	Thus
	$$ \abs{L_\epsilon} = \abs{\int_{-\infty}^0 g_\epsilon(y) (f(x - y) - f(x_+)) \, dy} $$
	Idea: $\lim_{y \to 0^+} f(x - y) - f(x_+) = 0$, so split into two parts, e.g. for some $y_0$
	$$ \abs{L_\epsilon} = \abs{\int_{-\infty}^{y_0} g_\epsilon(y) (f(x - y) - f(x_+)) \, dy + \int_{y_0}^0 \text{same}}
	\le \abs{\int_{-\infty}^{y_0} \cdots} + \abs{\int_{y_0}^0 \cdots} $$
\end{proof}

\section{French polynomials}

Exotic functions arise from solving PDEs in exotic geometric settings.

\begin{proposition}
	Assume ${p_n}_{n \in N}$ sequence of polynomials such that $\Deg p_n = n$,
	$p_0 \ne 0$.
	Then for each $k \in \mathbb N$, any polynomial of degree $k$ is a linear combination
	of ${p_j}_{j=0}^k$.
\end{proposition}

\begin{proposition}
	Let ${p_k}_{k=0}^\infty$ set of polynomials such that $\Deg p_k = k$, $p_0 \ne 0$.
	Moreover, assume that they are $L^2([a, b])$ orthogonal.
	Then these polynomials comprise an orthogonal basis of $L^2$ on $[a, b]$.
\end{proposition}
\begin{proof}
	Assume $f \in L^2[a, b]$ orthogonal to all these polynomials.
	Therefore by the preceding proposition, $f$ ortho to \emph{all} polynomials.
	Continuous functions are dense in $L^2$, i.e. given $\epsilon > 0$,
	there exists a continuous function, $g$, such that
	$$ \norm{f - g} < \epsilon $$
	Stone-Weierstrass Theorem says that all continuous functions on bounded intervals
	can be approximated by polynomials.
	Therefore, there exists a polynomial $p$ such that
	$$ \norm{g - p} < \epsilon $$
	Finally, using Cauchy-Schwarz inequality
	\begin{align*}
		\norm{f}^2 &= \langle f, f \rangle = \langle f - g + g - p + p, f \rangle
		= \langle f - g, f \rangle + \langle g - p, f \rangle + \cancel{\langle p, f \rangle} \\
		&\le \norm{f - g} \norm f + \norm{g - p} \norm f \\
		&\le 2 \epsilon \norm f
	\end{align*}
	Since $\epsilon > 0$ is arbitrary, this shows that $\norm f = 0$.
	Hence by the three equiv conds to be an ortho basis,
	the polynomials are an orthogonal basis of $L^2[a, b]$.
\end{proof}

\end{document}
