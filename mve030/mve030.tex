\documentclass{article}
\usepackage[T1]{fontenc}
\usepackage[utf8]{inputenc}
\usepackage[swedish]{babel}
\usepackage{amsmath, amsfonts, amsthm, mathtools}

\newtheorem{theorem}{Theorem}
\newtheorem{definition}{Definition}
\newtheorem{lemma}[theorem]{Lemma}
\newtheorem{proposition}{Proposition}
\newtheorem{corollary}{Corollary}[theorem]

\DeclarePairedDelimiter\abs{\lvert}{\rvert}
\newcommand{\norm}[1]{\left\lVert#1\right\rVert}

\title{MVE030: Fourier analysis}
\author{Axel Forsman}

\begin{document}
\maketitle

\begin{definition}{Periodicity}
	A function $f$ is \emph{periodic} with period $p$ if
	$$ f(x + p) = f(x) \quad \forall x \in \mathbb R $$
	and this is the smallest $p$ for which this holds.
\end{definition}

\begin{lemma}
	If $f$ is periodic with period $p$ then
	$\int_a^{a + p} f \, dx$
	is the same for all $a \in \mathbb R$.
\end{lemma}

\begin{proof}
	Let
	$$ q(a) \coloneqq \int_a^{a + p} f(x) \, dx = \int_0^{a + p} f \, dx - \int_0^a f \, dx $$
	Then the fundamental theorem of calculus gives $ q'(a) = f(a + p) - f(a) = 0 $.
\end{proof}

\begin{definition}{Hilbert space}
	A complete (every Cauchy sequence converges) normed vector space
	with a hermitian scalar product, that is

	$$ \langle f, g \rangle = \overline{\langle g, f \rangle} $$
\end{definition}

\begin{definition}{(Working definition of) $L^2([a, b])$}
	All functions $f : [a, b] \to \mathbb C$ such that $\norm f < \infty$.
	Hilbert space with scalar product
	$$ \langle f, g \rangle_{L^2} = \int\limits_a^b f(x) \overline{g(x)} \, dx $$
\end{definition}

\begin{proposition}{Standard estimate}
	Assume $\lvert f(x) \rvert \le M$ for (almost every) $x \in [a, b]$, then
	$$ \int_a^b \lvert f(x) \rvert^2 \, dx \le M^2 (b - a) $$
	thus $f \in L^2([a, b])$.
\end{proposition}
\begin{proof}
	$$ \int_a^b \lvert f(x) \rvert^2 \, dx \le \int_a^b M^2 \, dx = M^2 (b - a) \qedhere $$
\end{proof}

\begin{corollary}
	If $f$ is continuous on $[a, b]$ then $f \in L^2([a, b])$.
\end{corollary}

\begin{theorem}{Bessel's inequality}
	Let $H$ Hilbert space, $\{e_n\} \subset H$ orthonormal sequence, then
	$$ \sum_n \lvert \langle x, e_n \rangle \rvert^2 \le \norm{f}^2 \quad \forall f \in H $$
\end{theorem}

\begin{corollary}
	$$ \lim_{n \to \infty_-^+} c_n = 0 $$
\end{corollary}

\begin{definition}{Piecewise $C^k$}
	A function is piecewise $C^k$ on interval $I$ if it is $k$ times differentiable
	on $I \setminus S, \, S \coloneqq \{s_j\}_{j=1}^n$
	and $\forall j \, \exists \lim_{x \to {s_j}_-^+} f^{(j)}(x)$ for $l = 0, 1, \ldots, k$.
\end{definition}

\begin{theorem}{Pointwise convergence of Fourier series}
	Assume $f$ piecewise $C^1$ on $[-\pi, \pi]$ and $2\pi$ periodic on $\mathbb R$.
	Then $ S_N(x) \coloneqq \sum_{n = -N}^N c_n e^{inx} $ satisfies
	$$ \lim_{N \to \infty} S_N(x) = \frac{f(x_-) + f(x_+)}2 \quad \forall x \in \mathbb R $$
	When $f$ continuous at $x$ then $f(x_+) = f(x_-) = f(x)$.
\end{theorem}
\begin{proof}
	\begin{enumerate}
		\item Fix $x \in \mathbb R$
		\item Use definition of $c_n$:
			$ \displaystyle S_N(x) = \sum_{\lvert n \rvert \le N} \frac1{2\pi} \int\limits_{-\pi}^\pi f(y) e^{in (x - y)} \, dy $
		\item Change variables: Let $t \coloneqq y - x$. Then $y = t + x$, $dy = dt$
			\begin{align*}
				S_N(x) &= \sum_{\abs n \le N} \frac1{2\pi} \int\limits_{-\pi - x}^{\pi - x} \overbrace{f(t + x) e^{-int}}^{2\pi \text{ periodic}} \, dt \\
				&= \sum_{\abs n \le N} \frac1{2\pi} \int\limits_{-\pi}^{\pi} f(t + x) e^{-int} \, dt \\
				&= \int\limits_{-\pi}^{\pi} f(t + x) \left( \frac1{2\pi} \sum_{\abs n \le N} e^{-int} \right) \, dt
			\end{align*}
		\item Two facts about N\textsuperscript{th} Dirichlet kernel, $D_N(t) \coloneqq \frac1{2\pi} \sum_{-N}^N e^{\pm int}$:

			\begin{enumerate}
				\item $$ e^{int} + e^{-int} = 2 \cos nt \implies D_N(t) = \frac1{2\pi} \left(1 + \sum_{n = 1}^N 2\cos nt\right) $$
					$D_N$ is even: $\int_{-\pi}^0 D_N \, dt = \int_0^{\pi} D_N \, dt = \frac12$
				\item
					\begin{align*}
						D_N(t) &= \frac{e^{-iNt}}{2\pi} \sum_{n = 0}^{2N} e^{int} = \frac{e^{-iNt}}{2\pi} \frac{1 - e^{i(2N + 1)t}}{1 - e^{it}} \\
						&= \frac{e^{-iNt} - e^{i(N + 1)t}}{2\pi (1 - e^{it})}
					\end{align*}
			\end{enumerate}

			Our goal is to prove $\lim_{N \to \infty} \abs{S_N(x) - \frac{f(x_-) + f(x_+)}2} = 0$ or equivalently
			$\abs{\int_{-\pi}^\pi f(t + x) D_N(t) \, dt - \frac{f(x_-) + f(x_+)}2} \to 0, N \to \infty$
		\item $\frac12 f(x_-) = \int_{-\pi}^0 f(x_-) D_N(t) \, dt, \,
			\frac12 f(x_+) = \int_0^\pi f(x_+) D_N(t) \, dt$

			\begin{multline*}
				\abs{\int_{-\pi}^\pi f(t + x) D_N(t) \, dt - \int_{-\pi}^0 f(x_-)D_N(t) \, dt - \int_0^\pi f(x_+)D_N(t) \, dt} \\
				= \abs{\int_{-\pi}^0 (f(t + x) - f(x_-)) D_N(t) \, dt + \int_0^\pi (f(t + x) - f(x_+)) D_N(t) \, dt}
			\end{multline*}

		\item Substitute the second fact about $D_N$:
			\begin{multline*}
				\abs{\int_{-\pi}^0 \frac{e^{-iNt} - e^{i(N + 1)t}}{2\pi (1 - e^{it})} (f(t + x) - f(x_-)) \, dt \\
				+ \int_0^\pi \frac{e^{-iNt} - e^{i(N + 1)t}}{2\pi (1 - e^{it})} (f(t + x) - f(x_+)) \, dt}
			\end{multline*}

		\item Define
			$$ g(t) \coloneqq \begin{cases}
				\frac{f(t + x) - f(x_-)}{1 - e^{it}} & -\pi \le t < 0 \\
				\frac{f(t + x) - f(x_+)}{1 - e^{it}} & 0 \le t \le \pi
			\end{cases} $$
			\begin{align*}
				\lim_{t \to 0_-} \frac{f(t + x) - f(x_-)}{1 - e^{it}} &= \lim_{t \to 0_-} \frac{f(t + x) - f(x_-) t}{t (1 - e^{it})} \\
				= \frac{f'(x_-)}{-i e^{i(0)}} = \frac{f'(x_-)}{-i}
			\end{align*}
			where the second equality is due to $f$ being piecewise $C^1$.
			PSS $\lim_{t \to 0_+} g(t) = \frac{f'(x_+)}{-i}$
			Thus $g$ is piecewise $C^1$ too. Therefore $g$ is \emph{bounded} on $[-\pi, \pi]$.

			$$ \abs{S_N(x) - \frac{f(x_-) + f(x_+)}2}
			= \abs{\int_{-\pi}^\pi \left(\frac{e^{-iNt}}{2\pi} g(t) + \frac{e^{i(N + 1)t}}{2\pi} g(t) \right) \, dt} $$

		\item Bessel's inequality: For $g$ its $N$\textsuperscript{th} Fourier coefficient is
			$$ \frac1{2\pi} \int_{-\pi}^\pi g(t) e^{-iNt} \, dt \to 0, N \to \infty $$
			The $-N-1$\textsuperscript{st} Fourier coefficient is
			$$ \frac1{2\pi} \int_{-\pi}^\pi g(t) e^{i(N + 1)t} \, dt \to 0, N \to \infty \qedhere $$
	\end{enumerate}
\end{proof}

\begin{definition}{Fourier series}
	The Fourier series for a $2\pi$ periodic function is
	$$ \sum_{n \in \mathbb Z} c_n e^{inx} $$
\end{definition}

Want to differentiate/integrate solutions to PDE:s.

\begin{theorem}{On differentiation of Fourier series}
	Assume $f$ piecewise $C^1$ on $(-\pi, \pi)$, $2\pi$ periodic and continuous on $\mathbb R$.
	Then the Fourier coeffs $a_n, b_n, c_n$ and those of $f'$; $a'_n, b'_n, c'_n$; satisfy
	\begin{align*}
		a'_n &= n b_n & b'_n &= -n a_n & c'_n &= in c_n
	\end{align*}
\end{theorem}
\begin{proof}
	Use the definition of $c'_n$ and relate to $c_n$ through integration by parts
	\begin{align*}
		c'_n &\coloneqq \frac1{2\pi} \int_{-\pi}^\pi f'(x) e^{-inx} \, dx \\
		&= \frac1{2\pi} \left( \left. f(x) e^{-inx} \right\rvert_{-\pi}^\pi - \int_{-\pi}^\pi f(x) (-in) e^{-inx} \, dx \right) = in c_n
	\end{align*}
	where we have used that $f(x) e^{-inx}$ is $2\pi$ periodic.
\end{proof}

\begin{proposition}
	On $[-\pi, \pi]$, $\phi_n(x) \coloneqq \frac{e^{inx}}{\sqrt{2\pi}}, \, n \in \mathbb Z$ are orthonormal.
\end{proposition}

\end{document}
